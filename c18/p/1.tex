\startPROBLEM
(stacks on secondary storage)
在一臺計算機上,內存速度快,但量少,而磁盤速度慢,但量多,
如何實現\emph{棧}。
\ALGO{PUSH} 和 \ALGO{POP} 的操作對象爲單字。
我們希望這個棧可以比內存大得多,
其中大部分可以放在磁盤上。

一種簡單但低效的方法是將整個棧放在磁盤上。
在內存中僅保持一個指針,指向棧頂元素的磁盤地址。
如果該指針的值爲 $p$,
則棧頂元素是磁盤的第 $\lfloor \frac{p}{m}\rfloor$ 頁上的第 $p \mod m$ 個字,
這裏 $m$ 爲每頁所含字數。

爲了實現 \ALGO{PUSH},
我們可以增大棧指針,
將磁盤中對應頁讀到內存中,
將要入棧的元素複製到該頁上對應位置,
最後將該頁寫回磁盤。
 \ALGO{POP} 操作與之類似。
我們減小棧指針,從磁盤上讀入所需的頁,再返回棧頂元素。
由於沒有修改,所以無需寫回磁盤。

與分析 B 樹類似,
我們要統計兩部分代價:
總的磁盤存取次數和總的 CPU 時間。
從磁盤存取一個包含 $m$ 個字的頁面,
需要佔用 $\Theta(m)$ 的 CPU 時間。

% a
\startigBase[a]\startitem
從漸進意義上看,用這種簡單的方式實現,
在最壞情況下,
$n$ 個棧操作需要存取多少次磁盤?
CPU 時間又是多少?
(用 $m$ 和 $n$ 來表示這個問題及後面幾個問題的答案)
\stopitem\stopigBase

\startANSWER
$2n$ 次, $\Theta(mn)$。
\stopANSWER

現在考慮棧的另一種實現,
即在內存中始終保持存放棧中的一頁。
(還要用少量的內存來記錄當前哪一頁在內存中)
只有對應的磁盤頁面駐留在內存中,才能執行棧操作。
如果需要,可以將當前內存中的頁寫回磁盤,
並且可以從磁盤讀入新的一頁放入內存。
如果對應磁盤頁已經在內存中,
那麼就無需任何磁盤存取。

% b
\startigBase[continue]\startitem
最壞情況下, $n$ 個 \ALGO{PUSH} 操作所需的磁盤存取次數是多少?
所需的 CPU 時間是多少?
\stopitem\stopigBase

\startANSWER
$n/m$ 次, $\Theta(n)$。
\stopANSWER

% c
\startigBase[continue]\startitem
最壞情況下, $n$ 個棧操作所需的磁盤存取次數是多少?
所需的 CPU 時間是多少?
\stopitem\stopigBase

\startANSWER
$n/2$ 次, $\Theta(mn)$。
\stopANSWER

% d
假設還有一種實現方式,在內存中保持棧的 2 頁
(此外還有少量的字來記錄哪些頁在內存中)。
\startigBase[continue]\startitem
請描述如何管理棧頁,使得任何棧操作的攤還磁盤存取次數爲 $O(1/m)$,
攤還 CPU 時間爲 $O(1)$。
\stopitem\stopigBase

\startANSWER
少於 $m/3$ 時加載前面一頁。
多於 $2m/3$ 時加載後面一頁。
\stopANSWER

\stopPROBLEM