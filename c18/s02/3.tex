\startEXERCISE
Byunyan 教授斷言, \ALGO{B-TREE-INSERT} 所生成的樹高度總是最小的。
通過例子 $t=2$ 以及關鍵字集合 $\{1,2,...,15\}$ 說明
無論插入序列如何,所生成樹的高度都不可能最小,從而證明教授是錯的。
\stopEXERCISE

\startANSWER
該例 B 樹的高度最小為 2,除根節點外,每個節點 $x$ 關鍵字的數目 $1\le x.n\le 3$。

要讓樹的高度最小,那麼每個節點的關鍵字要儘可能多。
顯然,結果只能有一種,如下圖。

{\externalfigure[output/e18_2-3-1]}{}

假設現在只有最後一個節點沒有插入,
該關鍵字可能不能在根節點,
否則根節點少一個分支,最多只能容納 11 個關鍵字。

假設只剩 15 還沒有插入,如下圖:

{\externalfigure[output/e18_2-3-2]}{}

此時要插入 15,插入過程肯定要經過根節點,
而根節點已滿,必須分裂,
分裂後如下圖:

{\externalfigure[output/e18_2-3-3]}{}
此時無論如何也不可能變成第一張圖了,
這是由於 \ALGO{B-TREE-INSERT} 的預處理策略導致的。

進一步分析,第二張圖也是不可能產生的,
根節點在之前插入某個關鍵字後就分裂了。
\stopANSWER
