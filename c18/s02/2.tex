\startEXERCISE
請解釋在什麼情況下(如果有的話),
在調用 \ALGO{B-TREE-INSERT} 的過程中,
會執行冗餘的 \ALGO{DISK-READ} 或 \ALGO{DISK-WRITE} 操作。
(所謂冗餘的 \ALGO{DISK-READ},是指對已經在主存中的某頁做 \ALGO{DISK-READ}。
冗餘的 \ALGO{DISK-WRITE} 是指將已經存在於磁盤上某頁又完全相同的重寫一遍。)
\stopEXERCISE

\startANSWER
\ALGO{B-TREE-SPLIT-CHILD} 中,
第 1、 2 和 13 行分別對節點 $x$, $y$ 和 $z$ 進行更改,
第 18、 19 和 20 行對應的 \ALGO{DISK-WRITE} 操作,沒有冗餘。
主過程 \ALGO{B-TREE-INSERT} 中沒有調用 \ALGO{DISK-READ} 和 \ALGO{DISK-WRITE}。

\ALGO{B-TREE-SPLIT-ROOT} 中也沒有調用 \ALGO{DISK-READ} 和 \ALGO{DISK-WRITE}。

在 \ALGO{B-TREE-INSERT-NONFULL} 中,
由於第 3~7 行修改了 $x$,因此第 8 行的 \ALGO{DISK-WRITE} 不是冗餘的。
對 \ALGO{DISK-READ} 的唯一調用發生在第 12 行。
由於對 \ALGO{B-TREE-INSERT-NONFULL} 的調用是單程的,
因此沒有冗餘的 \ALGO{DISK-READ} 和 \ALGO{DISK-WRITE}。

綜上所述, \ALGO{B-TREE-INSERT} 利用預處理策略保證了其單程性質,
不會產生冗餘的讀寫操作。
\stopANSWER
