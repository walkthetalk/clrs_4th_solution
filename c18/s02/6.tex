\startEXERCISE
假設實現 \ALGO{B-TREE-SEARCH} 時,
在每個節點內採用二分查找,而不是線性查找。
證明:
無論怎樣選擇 $t$ ($t$ 爲 $n$ 的函數),
這種實現所需的 CPU 時間都爲 $O(\lg n)$。
\stopEXERCISE

\startANSWER
令 B 樹高爲 $h$,所含關鍵字數目爲 $n$,
則 $O(h)=O(\log_{t}n)$,
且 $h\le \log_{t}\frac{n+1}{2}$。
每個節點中關鍵字數目不會大於 $2t-1$,
二分查找用時 $O(\lg t)$。
因此總時間爲:
\startformula
O(\lg t \times \log_{t}n) = O(\lg t \times \frac{\lg n}{\lg t}) = O(\lg n)
\stopformula
與 $t$、 $n$ 之間的關係無關。
\stopANSWER