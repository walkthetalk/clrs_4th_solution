\startEXERCISE
因爲葉子節點無需指向孩子節點的指針,
那麼對於同樣大小的磁盤塊,可選用一個與內部節點不同的(更大的) $t$ 值。
請說明如何修改 B 樹的創建和插入過程來處理這個變化。
\stopEXERCISE

\startANSWER
最小度數爲 $t$ 的 B 樹,一個節點最多可存儲 $2t-1$ 個關鍵字以及 $2t$ 個指針。
而葉子節點沒有子節點,因此 $2t$ 個指針可以用來存儲關鍵字。
若將新的 $t$ 值記爲 $t'$,總空間要足夠容納 $2t'-1$ 個關鍵字,
即根據關鍵字和指針各佔空間多少,即可求出 $t'$ 的最大值。

B 樹的創建過程當無需修改(空節點)。
插入過程需要修改。
葉子節點分裂出來的節點仍是葉子節點,且仍只分裂出一個葉子節點,
不會出現一個葉子節點一次分裂出多個葉子節點的情況。
插入過程需要修改兩處:1)葉子節點和內部節點的分裂條件不同。
2)分裂後葉子節點和內部節點需要複製的關鍵字數目不同。

\CLRSH{GET-T(x)}
\startCLRSCODE
if x.leaf
	return t'
else
	return t
\stopCLRSCODE

\CLRSH{B-TREE-INSERT(T, k)}
\startCLRSCODE
r = T.root
if r.n == 2 \cdot \ALGO{GET-T(r)} - 1
	s = \ALGO{B-TREE-SPLIT-ROOT(T)}
	\ALGO{B-TREE-INSERT_NONFULL(s, k)}
else
	\ALGO{B-TREE-INSERT_NONFULL(r, k)}
\stopCLRSCODE

\CLRSH{B-TREE-INSERT-NONFULL(x, k)}
\startCLRSCODE
i = x.n
if x.leaf
	while i \ge 1 and k < x.key[i]
		x.key[i+1] = x.key[i]
		i = i - 1
	x.key[i + 1] = k
	x.n = x.n + 1
	\ALGO{DISK-WRITE(x)}
else
	while i \ge 1 and k < x.key[i]
		i = i - 1
	i = i + 1
	\ALGO{DISK-READ(x.c[i])}
	if x.c[i].n == 2 \cdot \ALGO{GET-T(x.c[i])} - 1
		\ALGO{B-TREE-SPLIT-CHILD(x, i)}
		if k > x.key[i]
			i = i + 1
	\ALGO{B-TREE-INSERT-NONFULL(x.c[i], k)}
\stopCLRSCODE

\stopANSWER