\startEXERCISE
請寫出 \ALGO{B-TREE-DELETE} 的僞碼。
\stopEXERCISE

\startANSWER
\CLRSH{B-TREE-MERGE(T, x, i)}
\startCLRSCODE
y = x.c[i]
z = x.c[i+1]

++y.n
y.key[y.n] = x.key[i]
for j = 1 upto z.n
	++y.n
	y.key[y.n] = z.key[j]
	if not y.leaf
		y.c[y.n] = z.c[j]
if not y.leaf
	y.c[y.n + 1] = z.c[z.n + 1]
free z

for j = i upto x.n - 1
	x.key[i] = x.key[i+1]
	x.c[i+1] = x.c[i+2]
--x.n

if x.n == 0 and T.root == x
	T.root = y
\stopCLRSCODE

\CLRSH{B-TREE-DELETE(T, x, k)}
\startCLRSCODE
i = 1
while i <= x.n and x.key[i] < k
	++i
if i <= x.n and x.key[i] == k
	if x.leaf
		delete x from x
	else
		y = x.c[i]
		z = x.c[i+1]
		if y.n >= T.t
			y' = y
			while not y'.leaf
				y' = y'.c[y'.n + 1]
			x.key[i] = y'.key[y'.n]
			B-TREE-DELETE(T, y, x.key[i])
		else if z.n >= T.t
			z' = z
			while not z'.leaf
				z' = z'.c[1]
			x.key[i] = z'.key[1]
			B-TREE-DELETE(T, z, x.key[i])
		else
			B-TREE-MERGE(T, x, i)
			B-TREE-DELETE(T, y, k)
else
	if x.leaf
		print not found k
	else
		if x.c[i].n == T.t - 1
			if i <= x.n and x.c[i+1].n >= T.t
				y = x.c[i]
				z = x.c[i+1]
				++y.n
				y.key[y.n] = x.key[i]
				if not y.leaf
					y.c[y.n+1] = z.c[1]
				x.key[i] = z.key[1]
				for j = 1 upto z.n - 1
					z.key[j] = z.key[j+1]
					if not z.leaf
						z.c[j] = z.c[j+1]
				if not z.leaf
					z.c[z.n] = z.c[z.n+1]
				--z.n
			else if i > 1 and x.c[i-1].n >= T.t
				y = x.c[i-1]
				z = x.c[i]
				++z.n
				for j = z.n downto 2
					z.key[j] = z.key[j-1]
					if not z.leaf
						z.c[j+1] = z.c[j]
				z.key[1] = x.key[i]
				if not z.leaf
					z.c[2] = z.c[1]
					z.c[1] = y.c[y.n+1]
				x.key[i] = y.key[y.n]
				--y.n
			else
				if i > x.n
					--i
				B-TREE-MERGE(T, x, i)
		B-TREE-DELETE(T, x.c[i], k)
\stopCLRSCODE
\stopANSWER
