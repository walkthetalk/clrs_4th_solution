\startEXERCISE
證明: \ALGO{CONNECTED-COMPONENTS} 處理完所有的邊後,
當且僅當兩個頂點在同一相集合時,他們才屬於相同的連通分量。
\stopEXERCISE

\startANSWER
若兩個頂點位於同一個連通分量中,則存在一條由邊構成的路徑連接它們。
若兩個頂點由一條邊連接,則在處理該邊時將它們放入同一集合中。
在過程的某個時刻,這條路徑中的每條邊都已被處理,
此時這條路徑上的所有頂點都將位於同一集合中。
現在假設兩個頂點 $u$ 和 $v$ 位於同一集合中。
從該集合的任意一個頂點開始擴散,
一定能擴散到一個包含 $u$ 和 $v$ 的連通分量,
必然有一條從 $u$ 到 $v$ 的邊路徑,
這意味著 $u$ 和 $v$ 必定位於同一個連通分量中。
\stopANSWER
