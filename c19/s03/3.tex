\startEXERCISE
給出一個操作序列,共含 $m$ 個 \ALGO{MAKE-SET}、 \ALGO{UNION} 和 \ALGO{FIND-SET}
(其中 $n$ 個是 \ALGO{MAKE-SET}),
當僅使用按秩合併,而不是用路徑壓縮時,所需時間為 $\Omega(m\lg n)$。
\stopEXERCISE

\startANSWER
\ALGO{UNION} 的結果是一棵高度爲 $\lg n$ 的樹。

第一步:執行 $n$ 次 \ALGO{MAKE-SET}。
每次 \ALGO{MAKE-SET} 所需時間為 $\Theta(1)$,
總時間為 $\Theta(n)$。
\startformula
\{x_1\},\{x_2\},\{x_3\},\cdots,\{x_n\}
\stopformula

第二步:執行 $n-1$ 次 \ALGO{UNION}。
由於單獨使用按秩合併,所以每次 \ALGO{UNION} 所需時間為 $\Theta(1)$,
總時間為 $\Theta(n)$。

第三步:執行 $m-2n+1$ 次 \ALGO{FIND-SET}。
如果每次都查找深度最大的節點,
則每次運行時間為 $\Omega(\lg n)$。
如果 $m\ge 3n$,則至少有 $m/3$ 次 \ALGO{UNION},
總時間為 $\Omega(m\lg n)$。

綜上,總運行時間為 $\Omega(m\lg n)$。
\stopANSWER
