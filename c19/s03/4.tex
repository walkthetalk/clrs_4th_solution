\startEXERCISE
假設想要增加一個操作 \ALGO{PRINT-SET(x)},
給定節點 $x$,打印出 $x$ 所在集合的所有成員,順序可以任意。
如何對一棵不相交集合森林的每個節點僅增加一個屬性,
使得 \ALGO{PRINT-SET(x)} 所花費的時間同 $x$ 所在集合元素的個數呈線性關係,
並且其他操作的漸進運行時間變。
這裏假設我們可在 $O(1)$ 時間內打印出集合的每個成員。
\stopEXERCISE

\startANSWER
為每個節點增加一個屬性 $next$,形成一個循環單鏈表,
可以通過 $next$ 訪問下一個對象。

\CLRSH{MAKE-SET(x)}
\startCLRSCODE
x.p = x
x.next = x
x.rank = 0
\stopCLRSCODE

\CLRSH{UNION(x,y)}
\startCLRSCODE
\ALGO{LINK(\ALGO{FIND-SET(x)}, \ALGO{FIND-SET(y)})}
\stopCLRSCODE

\CLRSH{LINK(x,y)}
\startCLRSCODE
if x.rank > y.rank:
	y.p = x
else:
	x.p = y
	if x.rank == y.rank:
		y.rank = y.rank + 1
p = x.next
q = y.next
x.next = q
y.next = p
\stopCLRSCODE

\CLRSH{FIND-SET(x)}
\startCLRSCODE
if x \ne x.p:
	x.p = \ALGO{FIND-SET(x.p)}
return x.p
\stopCLRSCODE

\CLRSH{PRINT-SET(x)}
\startCLRSCODE
\ALGO{PRINT(x)}
t = x.next
while t \ne x:
	\ALGO{PRINT(t)}
	t = t.next
\stopCLRSCODE
\stopANSWER
