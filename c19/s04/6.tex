\startEXERCISE\DIFFICULT
引理 19.13 的證明以縮放勢的單位來支配 $O(s)$ 項中隱藏的常數。
證明過程中為了更精確,
你需要改變勢函數的定義(19.7),
兩種情況都乘以一個常數(比如 $c$),
來支配 $O(s)$ 中的常數。
為了適應這個更新後的勢函數,
分析的其餘部分要做哪些必要的變化?附引理 19.13:

每個 \ALGO{FIND-SET} 的攤還代價是 $O(\alpha(n))$。
\stopEXERCISE

\startANSWER
設隱藏因子為 $c$,修改定義(19.7):

\startformula[x]
\phi_q(x) =\startmathcases
\NC c\cdot \alpha(n) \cdot x.rank \NC \text{若 $x$ 是根節點或 $x.rank = 0$} \NR
\NC c\cdot \left(\left(\alpha(n) - \ALGO{level(x)}\right)\cdot x.rank - \ALGO{iter(x)}\right) \NC \text{若 $x$ 不是根節點且 $x.rank \ge 1$} \NR
\stopmathcases\stopformula

令 $T(n)=O(s)$,即 $T(n)=\le cs$。

\startformula\startmathalignment
\NC T(n)-c(s-(\alpha(n)+2)) \NC \le cs - c(s-(\alpha(n)+2)) \NR
\NC \NC = c(\alpha(n)+2) \NR
\NC \NC = O(\alpha(n)) \NR
\stopmathalignment\stopformula
相關引理和推論中運用定義(19.7)的部分也要乘以 $c$。

\noindentation\emph{引理 19.8}:

對於每個節點 $x$ 和所有操作計數 $q$,有 $0\le \phi_q(x)\le c\cdot \alpha(n)\cdot x.rank$。

\noindentation\emph{引理 19.9}:

若 $x$ 不是根節點且 $x.rank \ge 1$,則 $\phi_q(x) < c \cdot \alpha(n) \cdot x.rank$。

\noindentation\emph{引理 19.10}:

如果 $x$ 不是根節點且第 $q$ 個操作是 \ALGO{LINK} 或 \ALGO{FIND-SET},
則在這個操作之後, $\phi_q(x)\le \phi_{q-1}(x)$。
此外,如 $x.rank\ge 1$,
並且 \ALGO{level(x)} 或 \ALGO{iter(x)} 是由於第 $q$ 個操作而發生了改變,
則 $\phi_q(x)\le \phi_{q-1}(x) - c$。

\stopANSWER
