\startEXERCISE
假定使用鏈表表示和加權合併啓發式策略,
請給出圖 19.3 所示操作序列所需運行時間的漸進緊確界,
附圖 19.3:
\inputsamedir{tbl19-3}
\stopEXERCISE

\startANSWER
執行了 $n$ 次 \ALGO{MAKE-SET},
每次運行時間是 $\Theta(1)$,合計 $\Theta(n)$。

執行了 $n-1$ 次 \ALGO{UNION},
每次 \ALGO{UNION} 需要修改規模較小的集合中所有鏈表對象的 set 指針,
由於規模較小的集合中鏈表對象只有一個,需要時間為 $\Theta(1)$,
還要考慮更新 tail 指針和表長度,
每次 \ALGO{UNION} 這些更新需要時間 $\Theta(1)$,合計 $\Theta(n)$。

綜上,總運行時間為 $\Theta(n)$。
\stopANSWER