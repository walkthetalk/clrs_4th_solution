\startEXERCISE
對定理 19.1 的整體證明進行改造,
使用鏈表表示和加權合併啓發式策略,
使得 \ALGO{MAKE-SET}、 \ALGO{UNION} 和 \ALGO{FIND-SET} 的攤還時間上界爲 $O(1)$,
以及 \ALGO{UNION} 的攤還時間上界爲 $O(\lg n)$。
附定理 19.1:

使用鏈表表示和加權合併啓發式策略,
\ALGO{MAKE-SET}、 \ALGO{UNION} 和 \ALGO{FIND-SET} 總調用次數數為 $m$,
其中 \ALGO{MAKE_SET} 的調用次數為 $n$,這些操作總時間為 $O(m+n\lg n)$。
\stopEXERCISE

\startANSWER
使用鏈表和加權合併的啟發式策略,
一個操作序列中如果共有 $m$ 個 \ALGO{MAKE-SET}、 \ALGO{UNION} 和 \ALGO{FIND-SET},
其中有 $n$ 個 \ALGO{MAKE-SET},
$l$ 個 \ALGO{UNION}($l < n$),
$m-n-l$ 個 \ALGO{FIND_SET}。

總運行時間為:
\startformula
\sum_{i=1}^{m}c_i = O(m-l+l\lg l) = O(m+l\lg l)
\stopformula
令 $\sum_{i=1}^{m}c_i\le c(m+l\lg l)$,每個操作的均攤代價:

\inputsamedir{tbl19.2-3}

總均攤代價為:
\startformula\startmathalignment
\NC \sum_{i=1}^{m}\hat{c_i} \NC = cn + c(m-n-l) + c(\lg n+1) \NR
\NC \NC =c(m+l\lg n) \NR
\NC \NC >c(m+l\lg l) \NR
\NC \NC \ge \sum_{i=1}^{m}c_i \NR
\stopmathalignment\stopformula
即總均攤代價能支付總的實際代價,成立。

所以 \ALGO{MAKE-SET} 和 \ALGO{FIND-SET} 的均攤時間為 $c=O(1)$,
\ALGO{UNION} 的均攤時間為 $c(\lg n + 1) = O(n\lg n)$。
\stopANSWER
