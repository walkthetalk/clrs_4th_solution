\startEXERCISE
使用鏈表表示和加權合併啓發式策略,
寫出 \ALGO{MAKE-SET}、 \ALGO{FIND-SET} 和 \ALGO{UNION} 的僞碼。
並指明你在集合對象和表對象中所使用的屬性。
\stopEXERCISE

\startANSWER
\CLRSH{MAKE-SET}
\startCLRSCODE
// S 具備屬性 key, set, next
S = \ALGO{CREATE-NODE()}

x.set = S
x.next = NIL
S.head = x
S.tail = x
S.size = 1

return S
\stopCLRSCODE

\CLRSH{FIND-SET(x)}
\startCLRSCODE
return x.set.head
\stopCLRSCODE

\CLRSH{UNION(x, y)}
\startCLRSCODE
S_1 = x.set
S_2 = y.set

if S_1 \ge S_2.size
	S_1.tail.next = S_2.head
	z = S_2.head
	while z \ne NIL
		z.set = S_1
		z = z.next
	S_1.tail = S_2.tail
	S_1.size = S_1.size + S_2.size
	return S_1

return UNION(y, x)
\stopCLRSCODE

\stopANSWER