\startEXERCISE
Gompers 教授猜想也許有可能在每個集合對象中僅使用一個指針,
而不是兩個指針(head 和 tail),
同時仍然保留每個鏈表元素的 2 個指針。
請說明教授的猜想是有道理的,
並通過描述如何使用一個鏈表來表示每個集合,
使得每個操作與本章中描述的操作有相同的運行時間,來加以解釋。
同時描述這些操作是如何工作的。
你的方法應該允許使用加權合併啓發式策略,
並與本節所描述的有相同效果。
(\hint 使用鏈表的尾作爲集合的代表。)
\stopEXERCISE

\startANSWER
集合只有一個屬性 tail,指向鏈表的尾部對象,
鏈表中每個對象都包含一個集合元素,
一個 tail 指針指向鏈表尾部對象,
一個 next 指針指向下一個對象。
除尾部對象外, 所有鏈表對象的 tail 都指向尾部對象,
而尾部對象的 tail 指向鏈表的集合對象。
為了訪問第一個鏈表對象,採用循環鏈表,
除尾部對象外,每個對象的 next 指針指向下一個鏈表對象,
尾部對象的 next 指向鏈表頭部對象。

給定對象 $x$,我們可以在 $O(1)$ 時間內找到尾部對象:
如果 $x.tail.tail == x$,則 $x$ 就是尾部對象,
否則尾部對象是 $x.tail$。
我們也可以在 $O(1)$ 時間內找到鏈表頭部對象:
如果 $x.tail.tail == x$,則 $x.next$ 就是頭部對象,
否則頭部對象是 $x.tail.next$。

\startcombination[2*2]
{\externalfigure[e19_2-5-1]}{a}
{\externalfigure[e19_2-5-2]}{b}
{\externalfigure[e19_2-5-3]}{c}
{}{}
\stopcombination
\stopANSWER