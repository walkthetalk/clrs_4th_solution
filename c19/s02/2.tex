\startEXERCISE[exercise:19.2-2]
給出下面程序的結果,
並給出該程序中 \ALGO{FIND-SET} 返回的值。
這裏使用加權合併啓發式策略的鏈表表示。

假定 $x_i$ 所屬集合 和 $x_j$ 所屬集合大小相同,
則 \ALGO{UNION(x_i, x_j)} 會將 $x_j$ 所在鏈表拼接到 $x_i$ 所在鏈表的尾部。
\stopEXERCISE

\startCLRSCODE
for i=1 to 16
	\ALGO{MAKE-SET(x[i])}
for i=1 to 15 by 2
	\ALGO{UNION(x[i], x[i+1])}
for i=1 to 13 by 4
	\ALGO{UNION(x[i], x[i+2])}
\ALGO{UNION(x[1], x[5])}
\ALGO{UNION(x[11], x[13])}
\ALGO{UNION(x[1], x[10])}
\ALGO{FIND-SET(x[2])}
\ALGO{FIND-SET(x[9])}
\stopCLRSCODE

\startANSWER
兩個 \ALGO{FIND-SET} 返回的都是第一個集合,即 $x_1$ 所在集合。

{
\txx
\inputsamedir{tbl19.2-2}
}
\stopANSWER