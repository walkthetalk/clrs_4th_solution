\startPROBLEM
(Off-line minimum)
\emph{脫機最小值問題}(off-line minimum problem)是使用 \ALGO{INSERT} 和 \ALGO{EXTRACT-MIN}
維護一個動態集合 $T$,其中的元素取自域 $\{1,2,\ldots,n\}$。
輸入是一個操作序列 $S$,包含 $n$ 個 \ALGO{INSERT} 和 $m$ 個 \ALGO{EXTRACT-MIN},
其中 $\{1,2,\ldots,n\}$ 中的每個關鍵字只被插入一次。
我們希望確定每個 \ALGO{EXTRACT-MIN} 返回的是哪個關鍵字。
也就是說,我們需要填充一個數組 $extracted[1:m]$,
對於 $i=1,2,\cdots,m$, $extracted[i]$ 就是由第 $i$ 次調用 \ALGO{EXTRACT-MIN} 所返回的關鍵字。
所謂“脫機”,是指可以在確定任何返回的關鍵字之前處理整個序列 $S$。

\startigBase[a]\startitem
在下面脫機最小值問題的實例中,每個 \ALGO{INSERT(i)} 用 $i$ 值來表示,
每個 \ALGO{EXTRACT-MIN} 用字母 $E$ 來表示:
\startformula
4,8,E,3,E,9,2,6,E,E,E,1,7,E,5
\stopformula
將正確的值填入數組 $extracted$。
\stopitem\stopigBase

\startANSWER
$4,3,2,6,8,1$
\stopANSWER

爲了設計一個算法來解決此問題,我們把序列 $S$ 劃分成若干個同構的子序列,
即將 $S$ 表示為:
\startformula
I_1,E,I_2,E,I_3,\ldots,I_m,E,I_{m+1}
\stopformula
這裏每個 $E$ 代表調用一次 \ALGO{EXTRACT-MIN},
每個 $I_j$ 代表連續多次調用 \ALGO{INSERT}(可能爲空)。
對於每個子序列 $I_j$,
開始時把由這些操作插入的關鍵字放入集合 $K_j$,
如果 $I_j$ 爲空,那麼 $K_j$ 也爲空。然後執行下面的程序:

\CLRSH{OFF-LINE-MINIMUM(m, n)}
\startCLRSCODE
for i = 1 to n
	// 確定 $j$,滿足 $i \in K[j]$
	if j \ne m + 1
		extracted[j] = i
		// 令 $l$ 是大於 $j$ 的最小值,使得存在 $K[l]$
		K[l] = K[j] \cup K[l]	// 並銷燬 $K[j]$
return extracted
\stopCLRSCODE

\startigBase[continue]\startitem
證明:由 \ALGO{OFF-LINE-MINIMUM} 返回的數組 $extracted$ 是正確的。
\stopitem\stopigBase

\startANSWER
可以使用歸納,分情況討論。
\stopANSWER

\startigBase[continue]\startitem
描述如何用並查集數據結構來實現高效的 \ALGO{OFF-LINE-MINIMUM}。
給出其最壞情況下運行時間的緊確界。
\stopitem\stopigBase

\startANSWER
$n\alpha(n)$。
\stopANSWER

\stopPROBLEM
