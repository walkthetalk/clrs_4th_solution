\startPROBLEM
(Tarjan’s offline least-common-ancestors algorithm)
在一棵有根樹 $T$ 中,
兩個節點 $u$ 和 $v$ 的\emph{最小公共祖先}
(leat common ancestor) $w$ 是節點 $u$ 和 $v$ 的共同祖先,
且深度最大。
在\emph{脫機最小公共祖先問題}(offline least-common-ancestors problem)中,
給定一棵有根樹 $T$ 及任一集合 $P=\{\{u,v\}\}$ ($T$ 中節點的無序對),
我們希望確定 $P$ 中每對節點的最小公共祖先。

爲了解決脫機最小公共祖先問題,
下面的過程通過調用 \ALGO{LCA(T.root)},
對 $T$ 進行遍歷。
假設在執行遍歷之前,每個節點被着色爲白色。

\CLRSH{LCA(u)}
\startCLRSCODE
\ALGO{MAKE-SET(u)}
\ALGO{FIND-SET(u)}.ancestor = u
for \text{$T$ 中節點 $u$ 的每個孩子 $v$}
	\ALGO{LCA(v)}
	\ALGO{UNION(u,v)}
	\ALGO{FIND-SET(u)}.ancestor = u
u.color = BLACK
for \text{每個節點 $v$ ($\{u,v\} \in P$)}
	if v.color == BLACK
		\text{\ALGO{print} $u$ 和 $v$ 的最小公共祖先是 \ALGO{FIND-SET(v).ancestor}}
\stopCLRSCODE

\startigBase[a]\startitem
證明:對每對 $\{u,v\}\in P$,第 10 行恰好只執行一次。
\stopitem\stopigBase

\startANSWER
令 $\le_{LCA}$ 是一個全序關係,
如果先執行 \ALGO{LCA(u)} 的第 7 行再執行 \ALGO{LCA(v)} 的第 7 行,則 $u \le_{LCA} v$。

當我們執行 \emph{for} 循環時,還未執行第 7 行。
這意味著 $v.color$ 仍然是白色。
因此,執行 \ALGO{LCA(u)} 時會跳過節點對 $\{u, v\}$。
但是,在執行 \ALGO{LCA(v)} 的 \emph{for} 循環時,
由於已經執行過 \ALGO{LCA(u)} 的第 7 行,
 $u.color$ 變成了黑色。
也就是說,執行 \ALGO{LCA(v)} 時,會處理節點對 $\{u, v\}$。

$\le_{LCA}$ 的順序可能不太明顯,這依賴於具體實現。
他取決於第 3 行 \emph{for} 循環遍歷子節點的順序。
也就是說,他不只是跟圖的結構有關。
\stopANSWER

\startigBase[continue]\startitem
證明:在調用 \ALGO{LCA(u)} 時,並查集數據結構的集合數等於 $T$ 中 $u$ 的深度。
\stopitem\stopigBase

\startANSWER
假設在調用 \ALGO{LCA} 之前是成立的,
我們來證明執行了 \ALGO{LCA} 之後這個性質仍然成立,
在 \ALGO{LCA} 最後並查集的數目增加了 1。

假設 $u$ 的深度是 $d$,
則在調用 \ALGO{LCA(u)} 前,並查集數目為 $d$(根據假設)。
第 1 行會使並查集數目變為 $d+1$。
第 4 行,會針對子節點 $u$ 調用 \ALGO{LCA},
此時,並查集數目 $d+1$ 正好是 $u$ 的深度,即假設仍然成立。
第 4 行之後,並查集數目變為 $d+2$,
而第 5 行又將使並查集數目變回 $d$,
即下一次迭代假設仍然成立。
循環結束後,並查集數目不再發生變化,即函數結束時並查集數目為 $d+1$。
\stopANSWER

\startigBase[continue]\startitem
證明:對於每對 $\{u,v\}\in P$, \ALGO{LCA} 能正確地輸出 $u$ 和 $v$ 的最小公共祖先。
\stopitem\stopigBase

\startANSWER
假設 $u$ 和 $v$ 的最小公共祖先是 $w$,
當運行 \ALGO{LCA(w)} 時, $u$ 所在子樹的根節點是 $w$ 的某個孩子,
 $v$ 所在子樹的根節點是 $w$ 的另一個孩子。
不失一般性,我們可以假設先處理包含 $u$ 的子樹。

當處理完包含 $u$ 的子樹後,
子樹中所有節點的 ancestor 都指向了 $w$,
且均為黑色。在 \ALGO{LCA(w)} 返回前,這些都不會改變。
但是 \ALGO{LCA(w)} 返回前會先調用 \ALGO{LCA(v)},
所以在 \ALGO{LCA(v)} 的第 8 行,因為 $u$ 是黑色的,
我們會處理 $(u, v)$。
而 $u$ 的 ancestor 仍然是 $w$,我們會將其打印出來,這也是 \ALGO{LCA} 的正確結果。
\stopANSWER

\startigBase[continue]\startitem
假設使用\insection[disjoint_set_forests] 中的並查集數據結構實現,試分析 \ALGO{LCA} 的運行時間。
\stopitem\stopigBase

\startANSWER
前兩行的運行時間是常數,
對於每個子節點,我們都會調用 \ALGO{LCA},
\ALGO{UNION} 所用時間也是常數,
\ALGO{FIND-SET} 則是逆 Ackerman 函數的攤還時間。
我們會檢查所有與 $u$ 相鄰的節點是否是黑色,
也就是說在第 8~10 行, $P$ 中的每個節點對最多也就檢查兩次,
總時間為 $O(|T|\alpha(|T|) + |P|)$。
\stopANSWER

\stopPROBLEM
