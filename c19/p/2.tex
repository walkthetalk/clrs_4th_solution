\startPROBLEM
(Depth determination)
在\emph{深度測定}(Depth determination)問題中,
我們通過以下三個操作來維護一個有根樹的森林 $\cal{F} = \{T_i\}$:

\ALGO{MAKE-TREE(v)}:創建一棵樹,只包含一個節點 $v$。

\ALGO{FIND-DEPTH(v)}:返回節點 $v$ 在樹中的深度。

\ALGO{GRAFT(r, v)}:使得節點 $r$ (假定他爲一棵樹的樹根)成爲節點 $v$ 的孩子
(假定他在另一棵樹中,但是他本身可能是、也可能不是一棵樹的根)

% a
\startigBase[a]\startitem
假設採用類似於不相交集合森林的樹表示:
 $v.p$ 是節點 $v$ 的父節點,
除了 $v$ 是根時 $v.p = v$ 的這種情況。
進一步假設,我們可以通過置 $r.p=v$ 來實現 \ALGO{GRAFT(r,v)},
並且可以通過沿着查找路徑上升至根,
返回一個除 $v$ 以外的節點樹來實現 \ALGO{FIND-PATH(v)}。
證明:一組 $m$ 個 \ALGO{MAKE-TREE}、 \ALGO{FIND-DEPTH} 和 \ALGO{GRAFT} 操作
序列的最壞情況運行時間是 \m{\Theta(m^2)}。
\stopitem\stopigBase

\startANSWER
\ALGO{MAKE-TREE} 和 \ALGO{GRAFT} 的運行時間都是 $O(1)$。
\ALGO{FIND-DEPTH} 的運行時間与深度成線性關係。
$m$ 個操作,最大深度為 $m/2$,
所以 \ALGO{FIND-DEPTH} 的運行時間是 $O(m)$。
總運行時間為 $O(m^2)$。

先創建 $m/3$ 棵樹,然後調用 $m/3$ 次 \ALGO{GRAFT} 將其串在一起。
最後針對最深的節點調用 $m/3$ 次 \ALGO{FIND-DEPTH},
每次調用所需時間至少為 $m/3$。
所以總時間為 $\Theta((m/3)^2) = \Theta(m^2)$。
因此最壞運行時間為 $\Theta(m^2)$。
\stopANSWER

通過使用按秩合併與路徑壓縮啓發式策略,
能減少最壞情況運行時間。
我們使用不相交集合森林 ${\cal{S}} = \{S_i\}$,
其中每個集合 $S_i$ (他本身是一棵樹)對應於一棵森林 $\cal{F}$ 中的樹 $T_i$。
然而,集合 $S_i$ 中的樹結構沒有必要對應於 $T_i$ 的樹結構。
實際上, $S_i$ 的實現並沒有記錄準確的父子關係,
但他允許我們確定 $T_i$ 中任意節點的深度。

關鍵的思想是維護每個節點 $v$ 的一個“僞距離” $v.d$,
他倍定義爲使得沿着從 $v$ 到他的集合 $S_i$ 的根的簡單路徑上的僞距離之和等於 $T_i$ 中節點 $v$ 的深度。
也就是說,如果從 $v$ 到他在 $S_i$ 的根的簡單路徑爲 $v_0,v_1,\ldots,v_k$(這裏 $v_0=v$ 並且 $v_k$ 是 $S_i$ 的根),
那麼節點 $v$ 在樹 $T_i$ 上的深度爲 $\sum_{j=0}^{k}v_j\cdot d$。

% b
\startigBase[continue]\startitem
給出 \ALGO{MAKE-TREE} 的一種實現。
\stopitem\stopigBase

\startANSWER
新集合只含有一個節點,深度只能是 0 且其父節點就是他自己。
這種情況下,集合和對應的樹是沒辦法區分的。

\CLRSH{MAKE-TREE(v)}
\startCLRSCODE
v = \ALGO{ALLOCATE-NODE()}
v.d = 0
v.p = v
return v
\stopCLRSCODE
\stopANSWER

% c
\startigBase[continue]\startitem
說明應如何修改 \ALGO{FIND-SET} 來實現 \ALGO{FIND-DEPTH}。
你的實現要採用路徑壓縮,
並且他的運行時間應與查找路徑的長度呈線性關係。
試確保你的實現能正確地更新僞距離。
\stopitem\stopigBase

\startANSWER
修改 \ALGO{FIND-SET},除了返回集合對象,同時返回其父節點的深度。
更新當前節點 $v$ 的偽距 $v.d$ (加上所返回的深度)。
由於遞歸的原因,運行時間並沒有變化,仍與查找路徑長度呈線性關係。
要實現 \ALGO{FIND-PATH},
從 $v$ 向上遞歸即可。

\CLRSH{FIND-SET(v)}
\startCLRSCODE
if v != v.p
	(v.p, d) = \ALGO{FIND-SET(v.p)}
	v.d = v.d + d
	return (v.p, v.d)
return (v, 0)
\stopCLRSCODE
\stopANSWER

% d
\startigBase[continue]\startitem
通過修改 \ALGO{UNION} 和 \ALGO{LINK} 來實現 \ALGO{GRAFT(r,v)},
用於合併 $r$ 和 $v$ 所屬集合。
請確保正確更新僞距離。
注意,
集合 $S_i$ 的根沒有必要是對應樹 $T_i$ 的根。
\stopitem\stopigBase

\startANSWER
要實現 \ALGO{GRAFT},需要找到 $v$ 的實際深度,
並將其加到 $r$ 所屬樹 $S_i$ 的根節點的偽距上。 

\CLRSH{GRAFT(r, v)}
\startCLRSCODE
(x, d_1) = \ALGO{FIND-SET(r)}
(y, d_2) = \ALGO{FIND-SET(v)}
if x.rank > y.rank
	y.p = x
	x.d = x.d + d_2 + y.d
else
	x.p = y
	x.d = x.d + d_2
	if x.rank == y.rank
		y.rank = y.rank + 1
\stopCLRSCODE
\stopANSWER

\startigBase[continue]\startitem
有一操作序列,內含 $m$ 個 \ALGO{MAKE-TREE}、 \ALGO{FIND-DEPTH} 和 \ALGO{GRAFT},
其中 \ALGO{MAKE-TREE} 有 $n$ 個。
請給出此操作序列最壞情況運行時間的緊確界。
\stopitem\stopigBase

\startANSWER
這三個操作與離散集合上的 \ALGO{MAKE}、 \ALGO{FIND} 和 \ALGO{UNION} 具有相同的漸進時間。
最壞運行時間是 $O(m\alpha(n))$。
\stopANSWER

\stopPROBLEM
