\startPROBLEM
(Viterbi algorithm)
語音識別時可以在有向圖上進行動態規劃。
爲有向圖 $G=(V,E)$ 中的邊打上標籤,
從而形成一個人說某種特定語言的形式化模型。
每條邊 $(u,v)\in E$ 所打聲音標籤 $\sigma(u, v)$,
來自有限聲音集 $\Sigma$。
圖中從特定頂點 $v_0\in V$ 開始的每條有向路徑
都對應此模型所能生成的一個語音序列。
對於一條有向路徑,我們定義其標籤爲路徑中邊上標籤的簡單連接。
\startigBase[a]\startitem
設計高效算法,對給定的有向圖 $G$ (邊上帶標籤)、
特定頂點 $v_0$ 及 $\Sigma$ 上的聲音序列
 $s=\langle \sigma_1,\sigma_2,\ldots,\sigma_k\rangle$,
返回 $G$ 中從 $v_0$ 開始的一條路徑,
 $s$ 爲該路徑的標籤(如果存在這樣的路徑)。
否則,算法返回 \ALGO{NO-SUCH-PATH}。
分析算法的時間複雜度(\hint 第 20 章中的概念可能會有幫助)。
\stopitem\stopigBase

\startANSWER
\CLRSH{VITERBI(G, s, v_0)}
\startformula\startmathalignment
\NC S(i,j) \NC = \bigvee_{k:(v_k,v_i)\in E \land \sigma(v_k,v_i)=\sigma_j} S(k,j-1) \NR
\NC S(0,0) \NC = 1 \NR
\NC S(i,0) \NC = 0 \qquad 1\le i \le n-1 \NR
\stopmathalignment\stopformula

\CLRSH{VITERBI(G, s, v_0)}
\startCLRSCODE
if s.length = 0
	return v_0

for (v_0,v_1) \in E
	if \sigma(v_0,v_1) = \sigma_1
		res = \ALGO{VITERBI}(G, (\sigma_2,\cdots,\sigma_k), v_1)
		if res \ne \ALGO{NO-SUCH-PATH}
			return (v_0,res)

return \ALGO{NO-SUCH-PATH}
\stopCLRSCODE

時間複雜度可能爲 $O(k |V|^2)$,待定。
\stopANSWER

現在,假定每條邊 $(u,v)\in E$ 都關聯一個非負概率 $p(u,v)$,
他表示從頂點 $u$ 開始,經過邊 $(u,v)$,
產生對應的聲音的概率。
任何頂點的出射邊的概率之和均爲 1。
一條路徑的概率定義爲路徑上邊的概率之積。
對於從 $v_0$ 開始的一條路徑,
我們可以將其概率看作從 $v_0$ 開始進行“隨機遊走”(random walk),
最後恰巧經過這條路徑的概率。
所謂“隨機遊走”,是指當位於頂點 $u$ 時,隨機選擇一條出射邊前進,
每條邊被選中的概率就是他所關聯的概率。

\startigBase[continue]\startitem
擴展(a)中的算法,
使得返回的路徑從 $v_0$ 開始、標籤爲 $s$,且概率最大。
分析算法的時間複雜度。
\stopitem\stopigBase

\startANSWER
\startformula\startmathalignment
\NC S(i,j) \NC = \max_{k:(v_k,v_i)\in E \land \sigma(v_k,v_i)=\sigma_j}
  \left{p(v_k,v_i) \times S(k,j-1)\right} \NR
\NC S(0,0) \NC = 1 \NR
\NC S(i,0) \NC = 0 \qquad 1\le i \le n-1 \NR
\NC R(i,j) \NC = a_{k:(v_k,v_i)\in E \land \sigma(v_k,v_i)=\sigma_j}
  \left{p(v_k,v_i) \times S(k,j-1)\right} \NR
\stopmathalignment\stopformula

上一題中,我們只須找到一條有效路徑即可,
但本題我們需要遍歷所有有效路徑,找到概率最大的,
因此運行時間比上一題要長,
但時間複雜度好像並沒什麼變化,待定。

\CLRSH{VITERBI'(G, s, v_0)}
\startCLRSCODE
if s.length = 0
	return {v_0, 1}

r.seq = \ALGO{NO-SUCH-PATH}
r.prob = 0
for (v_0,v_1) \in E
	if \sigma(v_0,v_1) = \sigma_1
		res = \ALGO{VITERBI'}(G, (\sigma_2,\cdots,\sigma_k), v_1)
		if res.prob > r.prob
			r.seq = {v_0, res.seq}
			r.prob = p(v_0,v_1) * res.prob

return res
\stopCLRSCODE
\stopANSWER

\stopPROBLEM
