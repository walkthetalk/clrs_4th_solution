\startPROBLEM
(Breaking a string)
某種字串處理語言可以將一個字符串拆分爲兩段。
由於此操作需要複製字串,
因此要花費 $n$ 個時間單位將長度爲 $n$ 的字符串拆分爲兩段。
如果要將一個字串拆分爲多段,
拆分的順序會影響所花費的總時間。
例如,假定將長度爲 20 的字串
在第 2 個、第 8 個以及第 10 個字符後進行拆分
(字符從左至右,從 1 開始編號)。
如果他按由左至右的順序進行拆分,則第一次拆分花費 20 個時間單位,
第二次拆分花費 18 個時間單位(在第 8 個字符處拆分 3~20 間的字串),
第三次拆分花費 12 個時間單位,共花費 50 個時間單位。
但如果從右至左進行拆分,所花時間爲 20、 10、 8 共 38 個時間單位。
還可以其他順序進行拆分,如,可以線在第 8 個字符處拆分(時間 20),
接着在左邊第 2 個字符處拆分(時間 8),
最後在右邊第 10 個字符處拆分(時間 12),總時間爲 40。

設計算法,對給定的拆分位置,確定拆分順序,使其代價最小。
更形式化地,
給定長度爲 $n$ 的字串 $S$ 和拆分點序列 $L[1:m]$,
計算拆分的最小代價,以及最優拆分序列。

\startANSWER
令 $c[i,j]$ 爲按 $L[i:j-1]$ 中所有位置拆分的最低開銷,
爲 $L$ 添加兩個元素 $L[0] = 1, L[m+1] = n+1$:
\startformula
c[i,j] = \startmathcases
\NC L[j] - L[i] + \min_{i<k<j}(c[i,k] + c[k,j]) \NC \text{如果 $j > i+1$} \NR
\NC 0 \NC \text{如果 $j = i+1$} \NR
\stopmathcases
\stopformula
$c[0,m+1]$ 即爲所求。
\stopANSWER

\stopPROBLEM
