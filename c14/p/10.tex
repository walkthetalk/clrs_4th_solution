\startPROBLEM
(Planning an investment strategy)
你利用所掌握的算法知識獲得了一份令人興奮的工作,
簽約獎金 1 萬美元。
你決定利用這筆錢進行投資,目標是 10 年後獲得最大回報。
你決定請投資經理 G.I.Luvcache 管理你的簽約獎金,
 Luvcache 所在公司提供了如下規則供你選擇。
該公司提供 $n$ 種投資,從 $1$~$n$ 編號。
在第 $j$ 年,第 $i$ 種投資的回報率爲 $r_{ij}$。
換句話說,如果你在第 $j$ 年第 $i$ 種投資投入 $d$ 美元,
那麼在第 $j$ 年底,你會得到 $d r_{ij}$ 美元。
回報率是有保證的,即未來 10 年每種投資的回報率均已知。
你每年只能做出一次投資決定。
在每年年底,你既可以將錢繼續投入到上一年選擇的投資種類中,
也可以轉移到其他投資中(轉移到已有的投資種類,或者新的投資種類)。
如果跨年時不做投資轉移,
需要支付 $f_1$ 美元的費用,
否則,需要支付 $f_2$ 美元的諮詢費,其中 $f_2 > f_1$。
諮詢費年底繳,不管你是投入、轉出單筆投資還是多筆投資,都是 $f_2$。

\startigBase[a]\startitem
如上所述,本題允許你每年將錢投入到多種投資中。
證明:如果每年都只能將所有錢投入到單筆投資中,存在最優投資策略
(記住最優投資策略只需最大化 10 年後的回報,
無需關心任何其他目標,如最小化風險)。
\stopitem\stopigBase

\startANSWER
不失一般性,我們假設最優投資方案 S 中,
第一年你進行了兩筆投資,其中 $k$ 佔 $d_1$ 美元,
 $m$ 佔 $d_2$ 美元。
在此方案中,前 $j$ 年投資方案都沒變。
如果 $r_{k1}+r_{k2}+\cdots+r_{kj} >= r_{m1}+r_{m2}+\cdots+r_{mj}$,
那麼我們可以在這 $j$ 年中將 $d_1+d_2$ 美元全部投 $k$,
而其他投資保持不變,結果顯然不會比 $S$ 差,
但投資的種類變少了。
按照這種方式,我們可以將每年的投資種類降爲 1。
\stopANSWER

\startigBase[continue]\startitem
證明:規劃最優投資策略問題具有最優子結構性質。
\stopitem\stopigBase

\startANSWER
子問題有兩種:
一種是不做投資轉移,也就是說不會產生諮詢費;
另外一種就是不管之前是怎麼投資的,完全重選,要付諮詢費。
\stopANSWER

\startigBase[continue]\startitem
設計最優投資策略規劃算法,分析算法時間複雜度。
\stopitem\stopigBase

\startANSWER
令 $P_i$ 爲第 $i$ 年拿到的利潤。
令 $S_i$ 爲第 $i$ 年選擇的項目。
令 $r[i,j]$ 爲第 $i$ 項目在第 $j$ 年的利潤率。
令 $R_j$ 爲第 $j$ 年利潤率最高的項目編號。
\startformula
P_i = \max(
(P_{i-1} - f_1) \times r[S_{i-1},i],
(P_{i-1} - f_2) \times r[R_i,i]
)
\stopformula

時間複雜度爲 $\Theta(n)$?主要是查找 $R_i$,待定。
\stopANSWER

\startigBase[continue]\startitem
假定投資公司在上述規則上又加入了新的限制條款,
在任何時刻你都不能在任何單一投資種類中投入 15000 美元以上。
證明:最大化 10 年回報問題不再具有最優子結構性質。
\stopitem\stopigBase

\startANSWER
加入新規則後,則需要考慮投資數額,
我們無法確定所要解決的子問題的數量。
\stopANSWER

\stopPROBLEM
