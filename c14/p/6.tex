\startPROBLEM
(Planning a company party)
一位總經理正在向 Stewart 教授諮詢公司聚會的計劃。
公司的內部結構關係是層次化的,
即員工按主管——下屬關係構成一棵樹,根節點爲總經理。
人事部按“宴會交際能力”爲每個員工打分,分值爲實數。
爲了使所有參加聚會的員工都感到愉快,
總經理不希望員工及其直接主管同時出席。

公司總經理向 Stewart 教授提供公司結構樹,
採用 10.3 節介紹的左孩子右兄弟表示法描述。
樹中每個節點除保存指針外,
還保存員工的名字和宴會交際評分。
設計算法,給出一個出席名單,使得宴會交際評分之和最大。
分析算法的時間複雜度。
\stopPROBLEM

\startANSWER
令 $D[i]$ 爲僅邀請 $i$ 及其下屬的最大得分。
令 $E[i]$ 爲僅邀請 $i$ 的下屬的最大得分。
則:
\startformula\startmathalignment
\NC E[i] \NC = \sum_{j\in C_i}D[j] \NR
\NC D[i] \NC = \max(E[i], a_i + \sum_{j\in C_i}E[j]) \NR
\stopmathalignment\stopformula

$D[1]$ 即爲所求。
由於 $D[i]$ 和 $E[i]$ 僅依賴於 $D[j]$ 和 $E[j]$,
其中 $j$ 爲 $i$ 的下屬,我們可以按樹中自下而上的順序進行計算。
如果我們知道 $i$ 的下屬序號均大於 $i$,
則可以按從 $n$ 到 $1$ 的順序計算 $E[i]$、 $D[i]$。

\CLRSH{PLAN-PARTY(A, C)}
\startCLRSCODE
for i = n downto 1
	E[i] = 0
	D[i] = 0
	foreach j ∈ C[i]
		E[i] = E[i] + D[j]
		D[i] = D[i] + E[j]
	D[i] = D[i] + A[i]
	if E[i] > D[i]
		D[i] = E[i]
\stopCLRSCODE

第 $i$ 次迭代中的循環需要 $2|C_i|+4$ 步,
總共需要時間 $4n + 2\sum_i|C_i|$。
除總經理外,其他員工都有一個直接領導,因此 $\sum_i|C_i|=n-1$。
因此總時間爲 $\Theta(n)$。
\stopANSWER
