\startPROBLEM[problem:14-3]
(Bitonic euclidean traveling-salesman problem)
在\emph{歐幾裏德旅行商}問題中,
給定平面上 $n$ 個點作爲輸入,
希望求出連接所有 $n$ 個點的最短巡遊路線。
圖 14-11(a) 給出了一個 7 點問題的解。
此問題是 NP-hard 問題,
因此大家認爲多項式時間內無法求解(參見第 34 章)。

J.L. Bentley 建議將問題簡化,
限制巡遊路線爲\emph{雙調巡遊}(bitonic tours),
即從最左邊的點開始,嚴格向右前進,直至最右邊的點,
然後掉頭嚴格向左前進,直至回到起始點。
圖 14-11(b) 給出了這 7 個點的最短雙調巡遊路線。
問題簡化後,多項式時間內就可以求解了。

設計一個 $O(n^2)$ 時間的最優雙調巡遊路線算法。
假設任何兩個點的 $x$ 坐標均不同,
且所有實數運算都花費單位時間。
(\hint 由左至右掃描,對巡遊路線的兩個部分分別維護可能的最優解。)
\stopPROBLEM

\startANSWER
\startcombination[2*1]
{\externalfigure[p14_3-1][width=.3\textwidth]}{a}
{\externalfigure[p14_3-2][width=.3\textwidth]}{b}
\stopcombination

先對所有點按橫坐標進行排序,
得到序列 $\langle p_1, p_2, \ldots, p_n\rangle$。
最左端的點爲 $p_1$,最右端的點爲 $p_n$。
定義最短雙調路徑爲 $B[i,j]$,
包括 $p_1$ 到 $p_j$ 中所有點,
其中 $1\le i \le j\le n$。
此路徑包含兩部分:
從 $p_i$ 到 $p_1$ 的向左子圖,
以及從 $p_1$ 到 $p_j$ 的向右子圖。
根據題意,需要計算 $b[n,n]$。

\startformula\startmathalignment
\NC B[1,2] \NC = |p_1 p_2| \NR
\NC B[i,j+1] \NC = B[i, j] + |p_{j} p_{j+1}| \qquad \text{如果 \m{i < j};} \NR
\NC B[j, j+1] \NC = \min_{k=1}^{j-1}\{B[k,j+1] + |p_k p_{j}|\} \NR
\stopmathalignment\stopformula
\stopANSWER
