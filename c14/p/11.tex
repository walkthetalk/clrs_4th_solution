\startPROBLEM
(Inventory planning)
Rinky Dink 公司是一家製造溜冰場冰面整修設備的公司。
這種設備每個月的需求量都在變化,因此公司希望設計一種策略來規劃生產,
需求雖然有波動,但是可預測的。
公司希望設計接下來 $n$ 個月的生產計劃。
對第 $i$ 個月,公司知道需求 $d_i$,
即該月能夠銷售出去的設備數量。
令 $D=\sum_{i=1}^{n} d_i$ 爲後 $n$ 個月的總需求。
公司僱傭的全職員工,可以提供一個月製造 $m$ 臺設備的勞動力。
如果公司希望一個月內製造多於 $m$ 臺設備,
可以僱傭額外的兼職勞動力,
並爲製造出來的每臺機器付出 $c$ 美元。
而且,如果在月末有設備尚未售出,
公司還要付出庫存成本。
保存 $j$ 臺設備的成本可描述爲一個函數 $h(j)$,
其中 $1\le j \le D$、 $h(j)\ge 0$,
且 $h(j)$ 單調遞增。

設計庫存規劃算法,在滿足所有需求的前提下最小化成本。
算法運行時間應爲 $n$ 和 $D$ 的多項式函數。
\stopPROBLEM

\startANSWER
令 $L_i$ 爲第 $i$ 月的產品剩餘; $C_i$ 爲第 $i$ 月的成本。
\startformula
C_i = \startcases
\NC h(L_{i-1}-d_i) \NC \text{如果 $L_{i-1}\ge d_i$;($L_i=L_{i-1}-d_i$)} \NR
\NC 0 \NC \text{如果 $d_i - m\le L_{i-1} \le d_i$;($L_i=0$)} \NR
\NC c\times (L_{i-1}+m-d_i) \NC \text{如果 $L_{i-1} < di - m$;($L_i=0$)} \NR
\stopcases
\stopformula

$C$ 依賴於 $L$, $L_i$ 依賴於 $L_{i-1}$。
\stopANSWER
