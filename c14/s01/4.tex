\startEXERCISE
修改 \ALGO{CUT-ROD} 和 \ALGO{MEMOIZED-CUR-ROD-AUX},
使得 \CLRSCODE{for} 循環執行次數最多爲 $\lfloor n/2\rfloor$,
而不是 $n$。
除此之外還需要修改什麼?對運行時間有什麼影響?附:

\CLRSH{MEMOIZED-CUT-ROD(p, n)}
\startCLRSCODE
r = \ALGO{NEW-ARRAY(0,n)}	// 用於記錄解
for i = 0 to n
	r[i] = -\infty
return \ALGO{MEMOIZED-CUT-ROD-AUX(p, n, r)}
\stopCLRSCODE

\CLRSH{MEMOIZED-CUT-ROD-AUX(p, n, r)}
\startCLRSCODE
if r[n] \ge 0	// 長度爲 $n$ 時已有解?
	return r[n]
if n == 0
	q = 0
else
	q = -\infty
	for i = 1 to n	// $i$ 是第一次截的位置
 		q = \max \{q, p[i] + \ALGO{MEMOIZED-CUT-ROD-AUX(p, n - i, r)}\}

r[n] = q	// 記錄長度爲 $n$ 的解
return q
\stopCLRSCODE
\stopEXERCISE

\startANSWER
除了直接修改 \CLRSCODE{for} 循環的次數外,
還需要修改比較條件,另外還需要修改 $q$ 的初始化。
關於最後一點,有兩個原因:首先循環可能根本就不執行,如果不想修改 $q$ 的初始化,
也可以在初始化 $r$ 時進行處理 $r[1] = p[1]$;
其次 循環中遞迴調用時 $n$ 最多爲 $n-1$,即 $1+(n-1)$ 的情況,
不會處理 $0+n$ 的情況,也就是說會漏掉 $n$ 的情況,
因此需要將 $q$ 初始化爲 $p[n]$ 參與比較。
爲簡化代碼,以下程序做了一定優化:

\CLRSH{MEMOIZED-CUT-ROD'(p, n)}
\startCLRSCODE
if n \le 0
	return 0

r = \ALGO{NEW-ARRAY(1,n)}	// 用於記錄解
for i = 1 to n
	r[i] = -\infty
return \ALGO{MEMOIZED-CUT-ROD-AUX'(p, n, r)}
\stopCLRSCODE

\CLRSH{MEMOIZED-CUT-ROD-AUX'(p, n, r)}
\startCLRSCODE
// $n$ 肯定大於 $0$
if r[n] \ge 0	// 長度爲 $n$ 時已有解?
	return r[n]

q = p[n]	\hfill // 修改點 1
for i = 1 to \lfloor n/2 \rfloor \hfill // 修改點 2
	q = \max \{q,
		\ALGO{MEMOIZED-CUT-ROD-AUX'(p, i, r)} \hfill // 修改點 3
		+ \ALGO{MEMOIZED-CUT-ROD-AUX'(p, n - i, r)}\}

r[n] = q
return q
\stopCLRSCODE

\stopANSWER
