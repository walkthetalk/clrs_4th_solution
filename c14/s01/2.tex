\startEXERCISE
下面描述的“貪心”策略,可以用來解決鋼條切割問題,
但並不能保證結果是最優的,請給出反例。

令鋼條的長度為 $i$,定義\emph{密度}為 $p_i / i$,即每英寸的價值。
貪心策略就是先將長度為 $n$ 的鋼條切下一段,長度爲 $i$,
其中 $1\le i \le n$,使其密度最高。
剩下的長度爲 $n-i$,用此策略繼續切割。
\stopEXERCISE

\startANSWER
\bTABLE[align=center]
\bTR \bTD 長度 $i$   \eTD \bTD 1 \eTD \bTD 2 \eTD \bTD 3 \eTD \bTD 4 \eTD\eTR
\bTR \bTD 價格 $p_i$ \eTD \bTD 1 \eTD \bTD 20\eTD \bTD 33\eTD \bTD 36\eTD\eTR
\bTR \bTD 密度 $p_i/i$ \eTD \bTD 1\eTD \bTD 10\eTD \bTD 11\eTD \bTD 9\eTD\eTR
\eTABLE

假設給定鋼條長度爲 $4$,
按貪心策略先切下長度爲 $3$ 的一段,價格爲 $33$,
剩下一段價格爲 $1$,總價爲 $34$。
但最優解是長度均爲 $20$ 的兩段,總價爲 $40$。

更一般的:

\bTABLE[align=center]
\bTR \bTD 長度 $i$   \eTD \bTD $l_0$ \eTD \bTD $l_1$ \eTD \bTD $l_2$ \eTD \eTR
\bTR \bTD 價格 $p_i$ \eTD \bTD $p_0$ \eTD \bTD $p_1$\eTD \bTD $p_2$\eTD \eTR
\bTR \bTD 密度 $p_i/i$ \eTD \bTD $d_0$\eTD \bTD $d_1$\eTD \bTD $d_2$\eTD\eTR
\eTABLE

令 $l_0 < l_1 < l_2$, $d_0<d_1<d_2$,且 $l_0 + l_2 = 2 l_1$。
則對於長度爲 $2l_1$ 的鋼條,貪心策略會將其截成長度爲 $l_2$ 和 $l_0$ 的兩段,
總價爲 $l_0 d_0 + l_2 d_2$。
當然也可以截成長度均爲 $l_1$ 的兩段,則總價會變爲 $2 l_1 d_1$。
若貪心策略所得不是最優解,則 $l_0 d_0 + l_2 d_2 < 2 l_1 d_1$。
即:

\startsplitformula\startmathalignment
\NC l_0 d_0 + l_2 d_2 \NC < 2 l_1 d_1 \NR
\NC l_0 d_0 + l_2 d_2 \NC < (l_0 + l_2) d_1 \NR
\NC l_0 d_0 + l_2 d_2 \NC < l_0 d_1 + l_2 d_1 \NR
\NC l_2 (d_2 - d_1)   \NC < l_0 (d_1 - d_0) \NR
\stopmathalignment\stopsplitformula

只要滿足如上不等式,貪心策略所得就不是最優解。
例如, $l_0=1,l_1=2,l_2=3$, $d_0=1,d_1=5,d_2=6$,
對於長度爲 $4$ 的鋼條,按貪心策略所得就不是最優解。
\stopANSWER
