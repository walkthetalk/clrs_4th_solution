\startEXERCISE
如前所述,使用動態規劃,我們首先求解子問題,
然後在子問題中進行選擇,以構造原問題的最優解。
 Capulet 教授認爲,我們不必爲了求原問題的最優解
總是把所有子問題全求出來。
他建議,在求矩陣鏈乘法問題的最優解時,
我們總可以在求解子問題\emph{之前}
選定 $A_i A_{i+1}\cdots A_j$ 的劃分位置 $A_k$
(選定的 $k$ 使得 $p_{i-1} p_k p_j$ 最小)。
請找出一個反例,證明這種貪心的方式可能只能找到次優解。
\stopEXERCISE

\startANSWER
錯誤在於最小開銷是指
最小左半開銷、最小右半開銷以及 $p_{i-1} p_k p_j$ 的和,
而不單單是 $p_{i-1} p_k p_j$ 最小。
例如 $[1x1][1x2][2x3]$。
\stopANSWER
