\startEXERCISE
對於矩陣鏈乘法問題,要確定最優代價,下面哪種方法效率更高?
第一種是窮舉所有可能的括號化方案,
對每種方案計算乘法運算次數,
第二種是運行 \ALGO{RECURSIVE-MATRIX-CHAIN}。
證明你的結論。
附:

\CLRSH{RECURSIVE-MATRIX-CHAIN(p,i,j)}
\startCLRSCODE
if i == j
	return 0
m[i,j] = \infty
for k = i to j-1
	q = \ALGO{RECURSIVE-MATRIX-CHAIN(p,i,k)}
		+ \ALGO{RECURSIVE-MATRIX-CHAIN(p,k+1,j)}
		+ p_{i-1}p_k p_j
	if q < m[i,j]
		m[i,j] = q
return m[i,j]
\stopCLRSCODE
\stopEXERCISE

\startANSWER
第二種方法效率更高。

考慮兩種方法都是如何對待子問題的。

在第一種窮舉法中,對於用於分割矩陣鏈的每個位置,
需要找出左、右兩部分各自的所有括號化方案,
然後觀察左右兩側所有可能的組合。
組合的數目是左右各自括號化方案數目的乘積。

而第二種方法會找到左右兩側各自的最優括號化方案,
然後將這兩種方案組合起來。
組合的數目是 $O(1)$。

實際運行時間:
 14.2 已經證明了窮舉法運行時間爲 $O(4^n / n^{3/2})$,
下面來證明第二種方法的運行時間爲 $O(n3^{n-1})$。

假設算法的第 1~2 行以及 6~7 行運行時間爲常數 $c$。
遞迴式如下:
\startformula
T(n) \le \startmathcases
\NC c \NC \text{n=1} \NR
\NC c = \sum_{k=1}^{n-1}(T(k) + T(n-k) + c) \NC \text{n>1} \NR
\stopmathcases
\stopformula
在 $n>1$ 時,將不等式改寫爲:
\startformula
T(n)\le 2\sum_{i=1}^{n-1}T(i) + cn
\stopformula
用代入法可以證明 $T(n)=O(n 3^{n-1})$。
\stopANSWER
