%p7-5
\startPROBLEM
(Stack depth for quicksort)
\insection[desc_quicksort] 中的 \ALGO{QUICKSORT} 對其自身遞迴調用了兩次。
在調用 \ALGO{PARTITION} 後, \ALGO{QUICKSORT} 分別對左右兩側數據進行了遞迴排序。
 \ALGO{QUICKSORT} 中的第二個遞迴調用並不是必須的。
我們可以用循環控制結構來代替他。
這一技術稱爲\emph{尾遞迴消除(tail-recursion elimination)},
好的編譯器都有這一功能。
利用這一技術將 \ALGO{QUICKSORT} 變形爲下面的 \ALGO{TRE-QUICKSORT}:

\CLRSH{TRE-QUICKSORT(A, p, r)}
\startCLRSCODE
while p < r
	// Partition and sort the low side.
	q = \ALGO{PARTITION(A, p, r)}
	\ALGO{TRE-QUICKSORT(A, p, q-1)}
	p = q + 1
\stopCLRSCODE

% a
\startigBase[a]
\item 證明: \ALGO{TRE-QUICKSORT(A, 1, A.length)} 能正確地對數列 $A$ 進行排序。
\stopigBase

\startANSWER
原 \ALGO{QUICKSORT} 將數列劃分成兩部分,並對其分別遞迴調用自身。
而這個版本所做的事情是一樣的,只是方式不同,
不是調用 \ALGO{TRE-QUICKSORT},
而是改變 $p$,繼續循環。
\stopANSWER

編譯器通常使用\emph{棧}來存儲遞迴過程中的相關信息,
包括每一次遞迴調用的參數。
最近調用的信息存在棧的頂部,最初調用的信息存在棧的底部;
開始調用時,則將其信息\emph{壓入}棧,
當調用結束時,其信息則被\emph{彈出}。
因爲我們假設數列參數是用指針來指示的,
所以每次過程調用只需要 $O(1)$ 的棧空間。
\emph{棧深度}是在一次計算過程中所用棧空間的最大值。
% b
\startigBase[a,continue]\startitem
請描述一個場景,使得輸入規模爲 $n$ 的數列時,
 \ALGO{TRE-QUICKSORT} 所用棧深度爲 $\Theta(n)$。
\stopitem\stopigBase

\startANSWER
如果 \ALGO{PARTITION} 返回的一直是 $r$,
則棧深度爲 $\Theta(n)$。
在有序數列上就會出現這種情況。
\stopANSWER

% c
\startigBase[a,continue]\startitem
修改 \ALGO{TRE-QUICKSORT},
使其最壞情況下棧深度爲 $\Theta(\lg{n})$,
並且能夠保持時間複雜度期望值仍爲 $O(n\lg{n})$。
\stopitem\stopigBase

\startANSWER
如果每次針對較大的子數列進行尾遞迴,則能滿足要求。

\CLRSH{TRE-QUICKSORT‘(A, p, r)}
\startCLRSCODE
while p < r
	q = \ALGO{PARTITION(A, p, r)}
	if q < (p + r) / 2
		\ALGO{TRE-QUICKSORT’(A, p, q-1)}
		p = q + 1
	else
		\ALGO{TRE-QUICKSORT’(A, q+1, r)}
		r = q - 1
\stopCLRSCODE
\stopANSWER

\stopPROBLEM
