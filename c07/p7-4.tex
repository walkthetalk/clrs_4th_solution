%p7-4
\startPROBLEM
(Stooge sort)
Howard 和 Fine 教授提出了一種排序算法,
並將其命名爲 Stooge 排序。

\CLRSH{STOOGE-SORT(A,p,r)}
\startCLRSCODE
if A[p] > A[r]
	\ALGO{SWAP(A[p],A[r])}
if p+1 < r
	k = \lfloor (r-p+1)/3 \rfloor	// 向下取整
	\ALGO{STOOGE-SORT(A,p,r-k)}	// 開頭 2/3
	\ALGO{STOOGE-SORT(A,p+k,r)}	// 最後 2/3
	\ALGO{STOOGE-SORT(A,p,r-k)}	// 再次開頭 2/3
\stopCLRSCODE
% a
\startigBase[a]\startitem
證明調用 \ALGO{STOOGE-SORT(A,1,n)} 可以完成對數列 $A[1:n]$ 的排序。
\stopitem\stopigBase

\startANSWER
對於 $n=1,2$,顯然成立。
遞迴時,可將數列分爲三部分: $A[p,p+k-1]$, $A[p+k,r-k]$ 以及 $A[r-k+1:r]$,
分別記爲 $a,b,c$。
則依次排序爲 $a\le b$, $b\le c$, $a\le b$。
最終可以保證 $a\le b\le c$。
\stopANSWER

% b
\startigBase[a,continue]\startitem
給出 \ALGO{STOOGE-SORT} 最壞運行時間的遞迴式
及其緊確漸進界($\Theta$)。
\stopitem\stopigBase

\startANSWER
$\Theta(n)=3\Theta(\frac{2n}{3}) + O(1)$。

根據主方法, $a=3,b=3/2,\epsilon=\log_{3/2}3$,
因此解爲 $\Theta(n^{\log_{3/2}3})$。
\stopANSWER

% c
\startigBase[a,continue]\startitem
比較 \ALGO{STOOGE-SORT}、插入排序、歸併配需、堆排序以及快速排序的最壞運行時間。
這些教授是否浪得虛名?
\stopitem\stopigBase

\startANSWER
$\log_{3/2}3\approx 2.71$。
論最壞運行時間,這種排序算法比其他所有算法都差。
插入排序是 $\Theta(n^2)$,
歸併排序是 $\Theta(n\lg n)$,
快速排序是 $\Theta(n^2)$,
堆排序是 $\Theta(n\lg n)$。
\stopANSWER

\stopPROBLEM
