\startEXERCISE\DIFFICULT
完成證明:
對於任意正奇數 $a$ 和任意整數 $r\ge 0$,
函數 $f_{a}^{(r)}$ 是一對一映射。
用反證法,並利用 $f_a$ 對 $2^w$ 取餘這個條件。
附:
\startformula
f_{a}(k)=swap((2k^2+ak)\mod 2^w)
\stopformula
\stopEXERCISE

\startANSWER
假設當 $r\ge 0$ 時,
 $\exists k\ne k': f_{a}^{(r)}(k)=f_{a}^{(r)}(k')$。
\startsplitformula\startmathalignment[n=3]
\NC f_{a}^{(k)} \NC = \NC f_{a}^{(r)}(k') \NR
\NC swap((2(f_{a}^{(r-1)}(k))^2 + a f_{a}^{(r-1)}(k))\mod 2^w)
   \NC =
   \NC swap((2(f_{a}^{(r-1)}(k'))^2 + a f_{a}^{(r-1)}(k'))\mod 2^w) \NR
\NC (2(f_{a}^{(r-1)}(k))^2 + a f_{a}^{(r-1)}(k))\mod 2^w
   \NC =
   \NC (2(f_{a}^{(r-1)}(k'))^2 + a f_{a}^{(r-1)}(k'))\mod 2^w \NR
\NC 2(f_{a}^{(r-1)}(k))^2 + a f_{a}^{(r-1)}(k)
   - 2(f_{a}^{(r-1)}(k'))^2 - a f_{a}^{(r-1)}(k') \NC \equiv \NC 0 (\mod 2^w) \NR
\NC (2(f_{a}^{(r-1)}(k) + f_{a}^{(r-1)}(k')) + a)(f_{a}^{(r-1)}(k) - f_{a}^{(r-1)}(k'))
   \NC \equiv \NC 0 (\mod 2^w) \NR
\stopmathalignment\stopsplitformula
由於 $a$ 爲正奇數,
 $(2(f_{a}^{(r-1)}(k) + f_{a}^{(r-1)}(k')) + a)$ 爲奇數,所以:
\startformula
(2(f_{a}^{(r-1)}(k) + f_{a}^{(r-1)}(k')) + a) \mod 2^w \ne 0
\stopformula
由同餘定理,對於同一個除數,兩個數的乘積與他們同餘的乘積同餘,所以:
\startformula
(f_{a}^{(r-1)}(k) - f_{a}^{(r-1)}(k')) \equiv 0 (\mod 2^w)
\stopformula
又當 $r\ge 1$ 時, $0\le f_{a}^{(r-1)}<2^w$,
所以 $f_{a}^{(r-1)}(k)=f_{a}^{(r-1)}(k')$。
重複以上步驟,可得 $f_{a}^{(0)}(k)=f_{a}^{(0)}(k')$,
即 $k=k'$,與 $k\ne k'$ 矛盾。
\stopANSWER