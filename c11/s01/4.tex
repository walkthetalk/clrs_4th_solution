\startEXERCISE\DIFFICULT
我們想在一個\emph{非常大}的數列上用直接尋址來實現一個字典。
即,如果數列大小爲 $m$,
則無論何時,字典中最多有 $n$ 個元素,
滿足 $m>>n$。
一開始,數列中可能包含一些無用信息,
並且由於他非常大,初始化整個數列也不切實際。
請給出你的方案。
所存儲的每個對象所佔空間爲 $O(1)$;
且 \ALGO{SEARCH}、 \ALGO{INSERT}
 和 \ALGO{DELETE} 所需時間也都應是 $O(1)$;
初始化數據結構的時間也應爲 $O(1)$。
(\hint 用一個附加數列,將其視爲棧,
其大小就是字典中實際存儲的關鍵字的數目,
用他來確定數列中的某一項是否有效。)
\stopEXERCISE

\startANSWER
只需將大數列中的元素和棧中的元素建立關聯,如:
棧中元素存儲數列元素的地址,而數列元素存儲棧上元素的索引。
 \ALGO{SEARCH} 時需要檢查上述關聯是否成立,若成立則有效,否則無效。
 \ALGO{INSERT} 對棧執行 \ALGO{PUSH};
 \ALGO{DELETE} 時則將棧當成普通數列,
如果要刪除的不是棧頂元素,則將棧頂元素轉移到所刪除的元素處,
並同步更新大數列中的相應元素。
\stopANSWER
