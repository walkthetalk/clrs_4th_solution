\startEXERCISE
通過將所有空閒槽位組成一個空閒鏈表,
如何爲散列表中的元素分配、釋放存儲空間?
假設一個槽位中可以存儲一個標記,
外加一個元素和一個指針,或者兩個指針。
所有字典操作、空閒鏈表操作的期望時間應爲 $O(1)$。
空閒鏈表是否有必要是雙向鏈表,
或者說單向空閒鏈表是否足夠?
\stopEXERCISE

\startANSWER
用這個標記來指示槽位是否空閒。

空閒槽位處於空閒鏈表中,
通過雙線鏈接將所有空閒槽位組成空閒鏈表。
空閒槽位中含有兩個指針。

而非空閒槽位中除了標記以外,
還含有一個元素值和一個指針(可能是 NIL),
這個指針指向散列到此槽位的下一個元素。
(當然,這個指針指向的其實是另一個槽位)

\emph{插入:}

如果新元素散列到空閒槽位上,
則將此槽位從空閒鏈表中移除,
將此元素的值填入,並將指針賦值爲 NIL。
爲了使其時間滿足 $O(1)$,
鏈表需爲雙向。

而如果新元素散列到了一個已用的槽位 $j$ 上,
則檢查此槽位原有元素 $x$ 是否應當散列到此槽位上。

如果是,則將新元素加入原有鏈表中即可。
需要先分配一個空閒槽位(即取出空閒鏈表的頭部節點),
填充數據,
並將此槽位加入槽位 $j$ 中的鏈表。

否則,新分配一個槽位,用來在 $j$ 所屬鏈表中替換 $j$。
並將 $j$ 作爲新鏈表的頭節點,填入新元素的值,並將指針賦值爲 NIL。
其中替換 $j$ 時,
需要從 $j$ 所屬鏈表的頭部順序查找,
才能找到 $j$ 的前驅節點。

\emph{刪除:}

令要刪除的元素爲 $x$,其散列到槽位 $j$ 上。

如果 $x$ 是 $j$ 中的元素,且 $j$ 沒有後繼節點,則釋放 $j$ 即可。

如果 $x$ 是 $j$ 中的元素,
但 $j$ 有後繼節點,則交換 $j$ 與其後繼節點的所有內容,
並釋放其後繼節點。

否則需要從 $j$ 開始,在鏈表中搜索 $x$,
找到後直接將其從鏈表中剔除,釋放即可。

\emph{查找:}

檢查關鍵字所散列到的槽位,
如果此槽位的原有元素也是散列到此位置上,
則在其鏈表中搜索(包括此槽位自身);
否則查找直接失敗。

所有操作期望時間均爲 $O(1)$。
在鏈表中查找的期望時間爲 $O(1+\alpha)$,
於鏈表的存儲位置無關,
所有元素都在表中則意味着 $\alpha\le 1$。
而如果空閒鏈表是單向的,
就無法在 $O(1)$ 時間內刪除任意槽位了。
\stopANSWER
