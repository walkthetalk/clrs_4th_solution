\startEXERCISE
Marley 教授提出了一個假設,
即改動鏈模式,
使每個鏈表都保持有序,
能有效提升散列的性能。
 Marley 教授的改動對成功的查找、
不成功的查找、插入和刪除操作的運行時間有何影響?
\stopEXERCISE

\startANSWER
對於成功的查找,
平均情況下的運行時間沒什麼變化,
還是 $\Theta(1+\alpha)$。
要檢查的元素數目比鏈表中小於目標的元素數目多一個。

不成功查找的平均時間也是 $\Theta(1+\alpha)$,
但是比成功查找要少一半。
因爲鏈表是有序的,
目標元素按數序所在位置是隨機的,
平均情況下要搜索鏈表元素的一半。

插入元素的平均時間爲 $\Theta(1+\alpha)$,
相對於原來的 $\Theta(1)$ 有所下降。
因爲鏈表需要保持有序。
與不成功查找的過程類似。

刪除元素與成功查找類似,
也是 $\Theta(1+\alpha)$。
\startformula
E[n_j] = \alpha = \frac{n}{2m}
\stopformula
\stopANSWER
