\startEXERCISE
假定數列 $Adj[u]$ 的每個記錄項不是鏈表,而是一個散列表,
裏面包含的是 $(u,v)\in E$ 的節點 $v$。
如果是均勻獨立散列且每條邊被查詢的概率相同,
則判斷一條邊是否在圖中的期望時間值是多少?
這種表示方法的缺陷是什麼?
請爲每條邊鏈表設計一個不同的數據結構來解決這個問題。
與散列表相比較,新方法存在什麼缺陷嗎?
\stopEXERCISE

\startANSWER
每個散列表越大,則期望時間越少,根據定理 11.2,判斷一條邊是否在途中的時間值為 $O(1+|E|/|V|)$,
當散列表跟頂點個數一樣大時,期望的查詢時間就是 $O(1)$ 了。

缺陷是空間佔用大(散列衝突)。

如果將邊鏈表改為紅黑樹,則時間值會降至 $O(1+\lg(|E|/|V|))$。
缺點是數據結構代價較大。

也可以用鄰接矩陣。
\stopANSWER
