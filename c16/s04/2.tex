\startEXERCISE
假定我們希望實現一個動態的開放地址散列表。
爲什麼我們要在裝載因子達到一個嚴格小於 1 的值 $\alpha$ 時就認爲表已經滿了?
簡要描述如何爲動態開放地址散列表設計一個插入算法,
使得每個插入操作的攤還代價期望值爲 $O(1)$。
爲什麼每個插入操作的實際代價期望值不必對所有插入操作都是 $O(1)$。
\stopEXERCISE

\startANSWER
根據定理 11.6 和推論 11.7,
向裝載因子爲 $\alpha$ 的開放尋址散列表插入一個關鍵字,
在平均情況下至多需要探測 $1/(1-\alpha)$ 次,
而 $\lim\frac{1}{1-\alpha}=+\infty$,
即當裝載因子無限趨近於 $1$ 時,
在開放尋址散列表中執行插入操作的期望開銷接近無窮大,
所以必須要考慮當裝載因子達到一個嚴格小於 $1$ 的值 $\alpha$ 時就認爲表滿了。

比如在 Java 中的 HashMap,源碼中的裝載因子缺省爲 $0.75$。

當 $\alpha_{i-1}=0.75$ 時,第 $i$ 個插入操作後,
表會擴展, $s_i = 2\cdot s_{i-1}$。

又比如 C++ 標準庫中的 unordered_set,
當 $n_{i-1}=s_{i-1}-1$ 時,
第 $i$ 個插入操作後,表進行擴展,
 $s_i$ 爲大於 $2\cdot s_{i-1}$ 的第一個質數。
注意, $T.num$ 對應 C++ 中的 $T.size()$,
 $T.size()$ 對應 C++ 中的 $T.bucket\_count()$。

模擬 Java 的啓發式策略,修改 \ALGO{TABLE-INSERT(T,x)} 的僞碼如下:

\CLRSH{TABLE-INSERT(T,x)}
\startCLRSCODE
if T.size == 0
	T.table = \ALGO{ALLOCATE(1)}
	T.size = 1
if T.num \ge T.size \cdot 3 / 4
	n = \ALGO{ALLOCATE(2\cdot T.size)}
	for i in T.table
		\ALGO{INSERT(n, i)}
	\ALGO{FREE(T.table)}
	T.table = n
	T.size = 2\cdot T.size
\ALGO{INSERT(T.table, x)}
T.num = T.num + 1
\stopCLRSCODE

此時每個插入操作的攤還代價期望值爲 $O(1)$,
但由於要考慮插入後表需要擴展的情況,
需要將舊表中的所有內容複製到新表中。
所以每個插入操作實際代價的期望不必對所有插入操作都是 $O(1)$。
\stopANSWER
