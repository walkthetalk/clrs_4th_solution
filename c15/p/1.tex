\startPROBLEM
(硬幣找零)
用最少的硬幣找零。
假設每枚硬幣的面值都是整數。

% a
\startigBase[a]\startitem
設計貪心算法求解找零問題,
硬幣面值爲 25 分、10 分、5 分、 1 分。
並證明其結果最優。
\stopitem\stopigBase

\startANSWER
將硬幣按面值從小到大排列: $d_1,d_2,\cdots,d_k$。

\CLRSH{MAKE-CHANGE(M)}
\startCLRSCODE
for i = k downto 1
	Q[i] = \lfloor M / d[i]\rfloor
	M = M - Q[i] * d[i]
\stopCLRSCODE
\stopANSWER

% b
\startigBase[continue]\startitem
假定硬幣面額是 $c$ 的冪,
即面額爲 $c^0,c^1,\ldots,c^k$,
其中 $c$ 和 $k$ 爲整數, $c>1$ 且 $k\ge 1$。
證明:貪心算法總能得到最優解。
\stopitem\stopigBase

\startANSWER
\startformula
M = \sum_{i=1}^k Q[i] \times d[i]
\stopformula

總的硬幣數目爲 $\sum_{i=1}^k n[i]$。
對於任意 $i < k$,都有 $Q[i] < c$。
否則,如果 $Q[i]\ge c$,
我們可以將 $Q[i]$ 減小 $c$,
並將 $Q[i+1]$ 增加 1。

下面來證明貪心算法會得到最優解。
令最優解中各面值硬幣數目爲 $O[i]$。
令 $j$ 第一個(最大的)滿足 $Q[j]\ne O[j]$。
我們知道 $Q[j]>O[j]$,
因爲貪心算法始終會曲最大數目的硬幣。
如果 $O[j]>Q[j]$,則最優解找零會超過 $M$。

\startformula\startmathalignment
\NC \sum_{i=1}^{j}O[i]\times d_i = \NC \sum_{i=1}^{j}Q[i]\times d_i \NR
\NC O[j]\times d_j + \sum_{i=1}^{j-1}O[i]\times d_i = \NC \sum_{i=1}^{j-1}Q[i]\times d_i + Q[j] \times d_j\NR
\NC \sum_{i=1}^{j-1}O[i]\times d_i = \NC \sum_{i=1}^{j-1}Q[i]\times d_i + (Q[j] -O[j]) \times d_j\NR
\NC \sum_{i=1}^{j-1}O[i]\times d_i = \NC \sum_{i=1}^{j-1}Q[i]\times d_i + (Q[j] -O[j]) \times c^{j-1}\NR
\NC \sum_{i=1}^{j-1}O[i]\times d_i - \sum_{i=1}^{j-1}Q[i]\times d_i = \NC (Q[j] -O[j]) \times c^{j-1} > c^{j-1}\NR
\stopmathalignment\stopformula

如果對於 \m{i=1,2,\ldots,j-1},滿足 \m{O[i] < c},則:
\startformula\startmathalignment
\NC     \NC \sum_{i=1}^{j-1}O[i]\times d_i \NR
\NC \le \NC \sum_{i=1}^{j-1}(c-1)\times c^{i-1} \NR
\NC   = \NC (c-1) \sum_{i=1}^{j-1} c^{i-1} \NR
\NC   = \NC (c-1) \sum_{i=0}^{j-2} c^i \NR
\NC   = \NC (c-1) \frac{c^{j-1}-1}{c-1} \NR
\NC   = \NC c^{j-1} - 1
\stopmathalignment\stopformula
這與上面得到的 \m{\sum_{i=1}^{j-1}O[i]\times d_i > c^{j-1}} 矛盾。因此最優解中編號爲 1 到 \m{j-1} 的硬幣中,必定有一種面值的硬幣個數達到或超過了 \m{c}。
但是如上所述,這樣的話就不可能是最優的了。

因此,貪心算法的解必定是最優解。
\stopANSWER

% c
\startigBase[continue]\startitem
設計一組硬幣面額,使得貪心算法不能保證得到最優解。
這組硬幣面額中應該包含 1 美分,
以保證任意 $n$ 都存在找零方案。
\stopitem\stopigBase

\startANSWER
面額爲 $10,9,1$。
要使得總面值爲 $27$,
貪心算法解爲 $2+7=9$ 個硬幣,
而最優解爲 $3$ 個硬幣。
\stopANSWER

% d
\startigBase[continue]\startitem
設計一個時間複雜度爲 $O(nk)$ 的找零算法,
使得無論 $k$ 種硬幣的面額爲多少,總能得到最優解(硬幣數最少),
假設肯定包含面額爲 $1$ 美分的硬幣。
\stopitem\stopigBase

\startANSWER
使用動態規劃。
令找零 $j$ 美分至少需要 $c[j]$ 枚硬幣。
\startformula
c[j] = \startcases
\NC 0 \NC \text{如果 $j\le 0$;} \NR
\NC 1 + \min_{1\le i \le k}\{c[j-d_i]\} \NC \text{如果 $j>1$。} \NR
\stopcases\stopformula
\stopANSWER


\stopPROBLEM
