\startEXERCISE
Gekko 教授計劃沿 U.S. 2 號高速公路用直排輪滑橫穿 North Dakota,
這條高速公路從 Minnesota 東部邊境的 Grand Forks 一直到達 Montana 西部邊境的 Williston。
教授能夠帶兩公升水,在喝光之前能滑行 $m$ 英里。
(North Dakota 相當平坦,教授無需擔心上坡路段喝水速度比平地或下坡路段快。)
教授從 Grand Forks 出發,並攜帶兩公升水。
地圖上顯示了高速公路上所有補給點,以及他們之間的距離。

教授的目標是儘量減少途中補充水的次數。
設計一個高效的方法,幫助教授確定應該在哪些地點補充水。
證明你的策略會生成最優解,並分析其運行時間。
\stopEXERCISE

\startANSWER
我們需要選擇補給站,
每次選擇的補給站離上一個的距離應最接近 $m$ 但不大於 $m$。

首先,此問題的解包含子問題的最優解。
令 $S$ 爲解, $G$ 爲選中的某個補給站。
如果從起點到 $G$ 之間,教授停下的次數不是最少,
則我們可以有一個更好的解。
但 $S$ 是最優解,所以 $S$ 包含子問題的最優解。

下面來證明貪心選擇會得到最優解。
令貪心選擇法選中的補給站爲 $G_1,G_2,\ldots,G_k$,
且這不是最優解。
則最優解中的第一個站點 $O_1$ 不是 $G_1$ 就是 $G_1$ 之前的站點,
否則從起點到 $O_1$ 的距離會超過 $m$ (與貪心策略矛盾)。
由於 $O_1$ 到 $O_2$ 的距離小於 $m$,
因此 $O_2$ 不是 $G_2$ 就是 $G_2$ 之前的站點。
以此類推, $O_k$ 不是 $G_k$ 就是 $G_k$ 之前的站點。
從而最優解選中站點個數不會少於貪心選擇法的解。
即,貪心選擇法的解就是最優解。

貪心選擇法的運行時間爲 $O_n$,其中 $n$ 爲站點總數。
\stopANSWER
