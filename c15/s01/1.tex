\startEXERCISE
根據遞迴式(15.2)爲作業選擇問題設計一個動態規劃算法。
算法應該按前文定義生成最大兼容作業集,並給出其大小 $c[i,j]$。
假定輸入的作業已按公式(15.1)排好序。
並與 \ALGO{GREEDY-ACTIVITY-SELECTOR} 比較一下運行時間。
附公式(15.1):
\startformula
f_1 \le f_2 \le f_3 \le \ldots \le f_{n-1} \le f_n
\stopformula

附遞迴式(15.2):
\startformula
c[i,j] = \startcases
\NC 0 \NC \text{如果 \m{S_{ij} = \Phi}} \NR
\NC \max_{a_k\in S_{ij}}\{c[i,k] + c[k,j] + 1\}
   \NC \text{如果 \m{S_{ij}\ne \Phi}} \NR
\stopcases\stopformula
\stopEXERCISE

\startANSWER
令 $S_{ij}$ 爲兩個不相重疊的作業 $a_i$ 和 $a_j$ 的間隔。
下列算法將爲作業選擇問題構建一張表,
其中 $s$ 和 $f$ 分別是作業的起始和結束時間,長度均爲 $n$。

\CLRSH{BUILD-ACTIVITY-TABLE(s, f)}
\startCLRSCODE
n = s.length
for i = 0 to n
	c[i][i] = 0

for m = 2 to n
	for i = 1 to (n-m+1)
		j = (i+m-1)
		c[i][j] = 0

		for k = i+1 to j-1
			if f[i] <= s[k] and f[k] <= s[j]
				q = (c[i][k] + c[k][j] + 1)
				if q > c[i][j]
					c[i][j] = q
					c[j][i] = k
return c
\stopCLRSCODE

此算法總運行時間爲 $O(n^3)$,而貪婪算法運行時間爲 $O(n)$。
\stopANSWER
