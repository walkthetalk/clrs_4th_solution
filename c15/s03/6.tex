\startEXERCISE
推廣 Huffman 算法,使之能生成三進制的碼字(即碼字由符號 0、 1、 2 組成),
並證明此算法能生成最優三進制碼。
\stopEXERCISE

\startANSWER
改成每次取頻率最小的三個節點組成一個新節點。
新節點的頻率爲三個子節點頻率之和。
但如果總數爲偶數,最後剩餘節點個數會小於 3,從而無法形成完整的三叉樹,
解決辦法是先增加一些頻率爲 0 的葉子節點,從而使得總數爲 $(k-1)t + 1$;
其中 $k$ 代表是 $k$ 叉樹,此時是 3, $t$ 爲任意正整數。
\stopANSWER
