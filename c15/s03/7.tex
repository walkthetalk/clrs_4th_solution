\startEXERCISE
假定一個數據文件由 8 位字符組成,
 256 個字符出現的頻率大致相同:
最高頻率低於最低頻率的 2 倍。
證明:在此情況下, Huffman 編碼並不比 8 位固定長度編碼更高效。
\stopEXERCISE

\startANSWER
令最小頻率爲 $f_{\min}$:
\startformula
B(T) = \sum_{c\in C}f(c) d_{T}(c)
     \ge \sum_{c\in C} f_{\min} \times d_{T}(c)
     = f_{\min}\times\sum_{c\in C}d_{T}(c)
\stopformula

其中 $\sum_{c\in C}d_{T}(c)$ 是葉子節點高度之和。
當 T 爲完全平衡二叉樹的時候,他取得最小值。
因此:
\startformula
B(T) \ge f\times \sum_{c\in C} d_{T}(c)
     \ge f\times 256 \times 9
\stopformula

而直接只用 8 位編碼的話,則:
\startformula
B(T') = \sum_{c\in C} 8 \times f(c)
      \le \sum_{c\in C} 8 \times 2 \times f
      = 256 \times 8 \times 2 \times f
\stopformula
\stopANSWER
