\startPROBLEM
(Small order statistics)
要找出 $n$ 個元素中的第 $i$ 個順序統計量,
令 \ALGO{SELECT} 在最壞情況所用比較次數爲 $S(n)$。
儘管 $S(n)=\Theta(n)$,
但 $\Theta$ 所隱含的常數因子非常大。
當 $i$ 相對 $n$ 比較小時,
可以 \ALGO{SELECT} 作爲子過程,
重新設計一個算法,以減少最壞情況下的比較次數。
\startigBase[a]\startitem
設計一個算法,找出 $n$ 個元素中第 $i$ 小的元素,
所用比較次數記爲 $U_i(n)$,滿足:
\startformula
U_i(n)=\startmathcases
\NC T(n) \NC \text{若 $i\ge n/2$,} \NR
\NC \lfloor n/2 \rfloor + U_i(\lceil n/2 \rceil) + T(2i) \NC \text{否則。} \NR
\stopmathcases
\stopformula
(\hint 開始先比較 $\lfloor n/2\rfloor$ 對互不相關的元素,
然後在每對元素中較小的數所組成的集合上進行遞迴)
\stopitem\stopigBase

\startANSWER
修改過的 \ALGO{SELECT},不但能找到第 $i$ 個順序統計量,
同時會劃分數列,找到 $i-1$ 個更小的元素。
\startigNum
\item 如果 $i\ge n/2$,直接調用 \ALGO{SELECT};
\item 否則,將數列劃分成一對對元素,每一對元素間進行比較;
\item 選取每一對元素中較小的那個,但是跟蹤另外一個(即保持成對的關聯);
\item 遞迴查找較小元素集合中的最小的 $i$ 個元素,
這樣連同每個元素所對應的元素,一共有 $2i$ 個元素;
\item 前 $i$ 個順序統計量肯定在這 $2i$ 個元素中,
在這 $2i$ 個元素上調用 \ALGO{SELECT}。結果即爲所求。
\stopigNum
其中第三步會將較小元素集合劃分成兩部分 $S_1$ 和 $S_2$,
在較大元素集合中分別對應 $B_1$ 和 $B_2$,
則 $S_1$ 和 $B_1$ 中的元素個數均爲 $i$,四個集合元素總數爲 $n$。
由於 $S_2$ 和 $B_2$ 中的元素均大於 $S_1$ 中的所有元素,
即 $S_2$ 和 $B_2$ 中的元素在所有 $n$ 個元素中排序的序號均大於 $i$。
所以前 $i$ 個順序統計量肯定在 $S_1$ 和 $B_1$ 中。
\stopANSWER

% b
\startigBase[continue]\startitem
證明:如果 $i<n/2$,則 $U_i(n)=n+O(T(2i)\lg(n/i))$。
\stopitem\stopigBase

\startANSWER
代入法:
\startsplitformula\startmathalignment
\NC U_i(n)
    \NC = \lfloor n/2 \rfloor + U_i(\lceil n/2 \rceil) + T(2i) \NR
\NC \NC = \lfloor n/2 \rfloor + \lceil n/2 \rceil +
             O(T(2i)\lg(\lfloor n/2 \rfloor / i)) + T(2i) \NR
\NC \NC = n + O(T(2i)\lg(n/i)) + T(2i) \NR
\NC \NC = n + O(T(2i)\lg(n/i)) \NR
\stopmathalignment\stopsplitformula
\stopANSWER

\startigBase[continue]\startitem
證明:如果 $i$ 是小於 $n/2$ 的常數,則 $U_i(n)=n+O(\lg{n})$。
\stopitem\stopigBase

\startANSWER
\startsplitformula\startmathalignment
\NC U_i(n)
    \NC = n + O(T(2i)\lg(n/i)) \NR
\NC \NC = n + O(O(1)\lg(n/i)) \NR
\NC \NC = n + O(\lg{n} - \lg{i}) \NR
\NC \NC = n + O(\lg{n} - O(1)) \NR
\NC \NC = n + O(\lg{n}) \NR
\stopmathalignment\stopsplitformula
\stopANSWER

% d
\startigBase[continue]\startitem
證明:如果 $i=n/k$,且 $k\ge 2$,則 $U_i(n)=n+O(T(2n/k)\lg{k})$。
\stopitem\stopigBase

\startANSWER
\startsplitformula\startmathalignment
\NC U_i(n)
    \NC = n + O(T(2i)\lg(n/i)) \NR
\NC \NC = n + O(T(2n/k)\lg(n/(n/k))) \NR
\NC \NC = n + O(T(2n/k)\lg{k}) \NR
\stopmathalignment\stopsplitformula
\stopANSWER

\stopPROBLEM
