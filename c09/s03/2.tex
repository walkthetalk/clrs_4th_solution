\startEXERCISE,
在 \ALGO{SELECT} 中第 1~10 行預處理時,
判斷條件改爲 $n\ge n_0$,
其中 $n_0$ 是一個常數;
令 $g$ 爲 $(\lfloor r-p+1)/5\rfloor$,
則 $A[5g+1:n]$ 中的元素不屬於任何一組。
證明:遞迴的運行時間可能會變得比較複雜,
但仍然是 $\Theta(n)$。
\stopEXERCISE

\startANSWER
分組後剩餘 $A[5g+1:n]$ 的元素個數爲 $0\le(n\mod 5)\le 4$。

先分析除遞迴調用外運行時間的上界,
跳過第 1~10 行,每次迭代開始時,
第 12~13 行 $5$ 個元素一組排序的運行時間爲 $\Theta(n)$。
第 17 行 \ALGO{PARTITION-AROUND} 運行時間也是 $\Theta(n)$。
剩餘的記錄操作運行時間爲 $\Theta(1)$。
所以遞迴調用外的操作運行時間爲
 $\Theta(n)+\Theta(n)+\Theta(1)=\Theta(n)$。

再分析遞迴調用的時間。
因爲 $g\le n/5$,
所以第 16 行選擇樞軸的遞迴調用運行時間 $T(g)\le T(n/5)$。
第 23、 24 行的遞迴調用指揮執行一個,
無論執行哪一個,遞迴調用的數據規模最多爲 $7n/10$,
因此第 23、 24 行的運行時間最多爲 $T(7n/10)$。
現在我們來證明樞軸的選擇方法保持了這一特性。
(爲每組找中位數並選擇所有中位數的中位數 $x$ 爲樞軸。)

圖 9.3 中的有向圖,右箭頭表示有向邊,
每條有向邊的起點元素小於或等於其終點元素,
黃色區域的元素大於或等於 $x$,
藍色區域的元素小於或等於 $x$。

黃色區域有 $3(\lfloor g/2\rfloor +1)\ge 3g/2$ 個元素,
藍色區域有 $3(\lceil g/2\rceil)\ge 3g/2$ 個元素,
黃色區域的元素不會落在小於或等於樞軸的一側,
藍色區域的元素不會落在大於或等於樞軸的一側,
所以排除其中一個顏色區域,剩餘元素最多爲:
\startformula
5g-3g/2+(n\mod 5)
  =7g/2+(n\mod 5)
  =7\lfloor n/5\rfloor / 2 + (n\mod 5)
  \le 7n/10 + 4
\stopformula
最壞情況下 \ALGO{SELECT} 運行時間的遞迴式:
\startformula
T(n)\le T(n/5) + T(7n/10+4) + \Theta(n)
\stopformula
由代入法,猜測 $T(n)=O(n)$,
即 $\exists c>0: \forall n>0$,當 $n\ge 5$ 時:
\startformula\startmathalignment
\NC T(n)
   \NC \le c(n/5) + c(7n/10+4) + \Theta(n) \NR
\NC\NC = cn - cn/10 + 4c + \Theta(n) \NR
\NC\NC \le cn \NR
\stopmathalignment\stopformula

當 $c$ 足夠大時,使得 $-cn/10+4c+\Theta(n)\le 0$,
 $T(n)\le cn$ 成立。

當 $n\le 4$ 時,爲基本情況,當 $c$ 足夠大時, $T(n)\le cn$ 成立。

因此 $T(n)=O(n)$,又第 13 行運行時間爲 $\Theta(n)$,
所以 \ALGO{SELECT} 運行時間爲 $\Theta(n)$。
\stopANSWER
