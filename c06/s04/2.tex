\startEXERCISE
試分析在使用下列循環不變式時, \ALGO{HEAPSORT} 的正確性:

在算法的第 2~5 行 \emph{for} 循環每次迭代開始時,
子數列 $A[1..i]$ 是一個最大堆,包含了數列 $A[1..n]$ 中最小的 $i$ 個元素,
而子數列 $A[i+1..n]$ 則包含了數列 $A[1..n]$ 中已排序的 $n-i$ 個最大元素。
\stopEXERCISE

\startANSWER
\emph{初始化:}子數列 $A[i+1..n]$ 爲空,不變式成立;

\emph{保持:} $A[1]$ 是 $A[1..i]$ 中最大的,但是小於 $A[i+1..n]$ 中所有元素。
將 $A[1]$ 和 $A[i]$ 調換後,則 $A[i..n]$ 中的元素是最大的,且是排好序的。
堆的大小減一,並調用 \ALGO{MAX-HEAPIFY} 會將 $A[1..i-1]$ 構造成最大堆。
將 $i$ 減一,繼續下一次迭代;

\emph{終止:}待 $i=1$ 時, $A[2..n]$ 是排好序的,而 $A[1]$ 是最小的,
因此整個數列爲排好序的。
\stopANSWER
