\startPROBLEM
(用插入的方式建堆 Building a heap using insertion)
我們可以通過反復調用 \ALGO{MAX-HEAP-INSERT} 不斷向堆中插入元素來構建一個堆。
考慮 \ALGO{BUILD-MAX-HEAP} 如下實現方式
(假設所插入對象就是堆元素):

\CLRSH{BUILD-MAX-HEAP'(A, n)}
\startCLRSCODE
A.size = 1
for i = 2 to n
	\ALGO{MAX-HEAP-INSERT(A, A[i], n)}
\stopCLRSCODE

% a
\startigBase[a]\startitem
當輸入數據相同時, \ALGO{BUILD-MAX-HEAP} 和 \ALGO{BUILD-MAX-HEAP'} 所生成的堆是否相同?
如果是,請證明;否則請舉出反例。
\stopitem\stopigBase

\startANSWER
不同。例如,輸入數據爲 $\langle 1, 2, 3, 4, 5, 6 \rangle$ 時,兩個堆分別如下:

\startcombination[2*1]
{\externalfigure[p6_1_a-1]}{}
{\externalfigure[p6_1_a-2]}{}
\stopcombination
\stopANSWER

% b
\startigBase[continue]\startitem
證明:調用 \ALGO{BUILD-MAX-HEAP'} 建立包含 $n$ 個元素的堆,
最壞情況下其時間復雜度爲 $\Theta(n\lg{n})$。
\stopitem\stopigBase

\startANSWER
最壞情況下, \ALGO{MAX-HEAP-INSERT} 的運行時間爲 $\Theta(\lg{n})$,會被調用 $n-1$ 次。
最壞情況下, \ALGO{MAX-HEAP-INSERT} 會將所有元素都移到堆的根節點上, 即需要 $\lg{k}$ 次交換,
無論 $k$ 的值是多少。
如果輸入數列已經是排好序的,則就是最壞情況。時間復雜度爲(參見\inexercise[lg_n_fac]):
\startformula
\sum_{i=2}^{n}\lg{i} = \lg(n!) = \Theta(n\lg{n})
\stopformula
\stopANSWER

\stopPROBLEM
