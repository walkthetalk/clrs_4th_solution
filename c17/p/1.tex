\startPROBLEM
(Point of maximum overlap)
假設我們希望記錄一個區間集合的\emph{最大重疊點},即覆蓋此點的區間數目最多。
\startigBase[a]\startitem
證明:總有一個最大重疊點是某個區間的端點。
\stopitem\stopigBase

\startANSWER
任選一個最大重疊點(如是任何區間的端點),
我們總能在所有區間端點中找到離他最近的那個端點,
覆蓋這個端點的區間數與我們所選重疊點是一樣的。
\stopANSWER

\startigBase[continue]\startitem
設計一個數據結構,使其能有效地支持 \ALGO{INTERVAL-INSERT}、 \ALGO{INTERVAL-DELETE},
以及 \ALGO{FIND-POM} (返回最大重疊點) 操作。
(\hint 用紅黑樹記錄所有的端點。
左端點關聯 +1,右端點關聯 -1,
並且給樹中的每個節點加些額外信息來維護最大重疊點。)
\stopitem\stopigBase

\startANSWER
令 $e_1,e_2,\cdots,e_n$ 爲已排序的所有端點,
滿足 $e_1\le e_2 \le \cdots \le e_n$。
從左至右掃描,若爲左端點,則 p 的值爲 +1,若爲右端點,則 p 的值爲 -1。
令 $s(i,j)=p_i + p_{i+1} + \cdots + p_j$,
其中 $1\le i\le j\le n$。
目標是要找到使 $s(1,j)$ 最大的 $j$。

每個節點記錄三個值:
\startigBase
\item $v$,子樹所有節點的 $p$ 之和;
\item $m$,子樹上 \m{s(i,j)} 的最大值;
\item $j$,即 $m$ 所對應的 $j$。
\stopigBase
定義 NIL 的 $m$ 和 $v$ 均爲 0。
用自底向上的方式計算這些值。
\startformula
x.m = \startcases
\NC x.left.m	\NC (x.j = x.left.j) \NR
\NC x.left.m + x.v \NC (x.j = x) \NR
\NC x.left.m + x.v + x.right.m \NC (x.j = x.right.j) \NR
\stopcases
\stopformula

\ALGO{FIND-POM} 直接返回 $T.root.j$ 即可。
\stopANSWER

\stopPROBLEM
