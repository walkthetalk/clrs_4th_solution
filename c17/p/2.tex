\startPROBLEM
(Josephus permutation)
此問題定義如下:假設有 $n$ 個人,編號分別爲 $1,2,3,\ldots,n$,按順序圍成一圈。
給定一個正整數 $m$ 滿足 $m\le n$。
從某個指定的人開始報數,數到 $m$ 的那個人出列;
下一個人又從 1 開始報數,
以此類推,直到所有人全部出列。
按出列的次序將所有人的編號進行排列,
稱其爲 \emph{(n,m)-Josephus 排列},
例如, (7,3)-Josephus 排列爲 $\langle 3,6,2,7,5,1,4\rangle$。

%a
\startigBase[a]\startitem
假設 m 是常數,描述一個 $O(n)$ 時間的算法,
使得對於給定的 n,能夠輸出 (n,m)-Josephus 排列。
\stopitem\stopigBase

\startANSWER
用循環鏈表即可,總時間爲 $O(mn)$,
由於 m 是常數,有 $O(mn)=O(n)$。
\stopANSWER

%b
\startigBase[continue]\startitem
假設 m 不是常數,描述一個 $O(n\lg n)$ 時間的算法,
使得對於給定的 n,能夠輸出 (n,m)-Josephus 排列。
\stopitem\stopigBase

\startANSWER
使用順序統計樹。
順序統計樹相關操作的時間爲 $O(\lg n)$,
總時間爲 $O(n\lg n)$。

\CLRSH{JOSEPHUS(n,m)}
\startCLRSCODE
T = NIL
for j = 1 to n
	x = \ALGO{NEW-NODE()}
	x.key = j
	\ALGO{OS-INSERT(T, x)}
j = 1
for k = n downto 1
	j = (j + m - 2) \mod k + 1
	x = \ALGO{OS-SELECT(T.root, j)}
	\ALGO{PRINT(x.key)}
	\ALGO{OS-DELETE(T, x)}
\stopCLRSCODE
\stopANSWER

\stopPROBLEM
