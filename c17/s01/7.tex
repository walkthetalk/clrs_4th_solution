\startEXERCISE
說明如何在 $O(n\lg n)$ 時間內,
利用順序統計樹計算數列(規模爲 n)中逆序對的數目(參見問題 2-4)。
\stopEXERCISE

\startANSWER
令數列爲 $A[1:n]$,將 $A[1],A[2],\dots,A[n]$ 依次插入一棵初始爲空的順序統計樹 T,
如果 $i < j$,當插入 $A[j]$ 時, $A[i]$ 已經在樹中。
如果 $A[j]$ 位於 $A[i]$ 的左子樹,即 $A[j]<A[i]$,
就產生了一個逆序對,即每次插入左子樹都會使逆序對數目增加 1。

爲 T 增加屬性 $T.inv$,初始 $T.inv = 0$。

\CLRSH{COUNT-INVERSIONS(A,n)}
\startCLRSCODE
T = \ALGO{NEW-ORDER-STATISTIC-TREE()}
T.inv = 0
for i = 1 to n
	z = \ALGO{NEW-NODE()}
	z.key = A[i]
	\ALGO{RB-INSERT(T, z)}
return T.inv
\stopCLRSCODE

修改 \ALGO{RB-INSERT(T,z)}:
\startCLRSCODE
x = T.root
y = T.nil
while x \ne T.nil
	y = x
	if z.key < x.key
		x = x.left
		T.inv = T.inv + 1
	else
		x = x.right
z.p = y
if y == T.nil
	T.root = z
elseif z.key < y.key
	y.left = z
else
	y.right = z
z.left = T.nil
z.right = T.nil
z.color = RED
\ALGO{RB-INSERT-FIXUP(T,z)}
\stopCLRSCODE

每次 \ALGO{RB-INSERT} 的運行時間爲 $O(\lg n)$,
總計插入 $n$ 個元素,所以 \ALGO{COUNT-INVERSIONS} 的運行時間爲 $O(n\lg n)$。
\stopANSWER
