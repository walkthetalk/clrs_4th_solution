\startEXERCISE
說明如何來維護一個動態集合 $Q$,使其支持操作 \ALGO{MIN-GAP},
該操作能給出 $Q$ 中兩個最接近的數之間的差值。
例如, $Q=\{1,5,9,15,18,22\}$,
則 \ALGO{MIN-GAP} 返回 $18-15=3$,
因爲 15 和 18 是 $Q$ 中兩個最接近的數。
同時要使得操作 \ALGO{INSERT}、 \ALGO{DELETE}、 \ALGO{SEARCH} 和 \ALGO{MIN-GAP} 儘可能高效,
並分析他們的運行時間。
\stopEXERCISE

\startANSWER
以紅黑樹作爲基礎數據結構,
並爲每個節點 $x$ 增加屬性:
\startigBase
\item min\text{-}gap:以 $x$ 爲根的子樹中兩個最接近的數之間的差值。如果 $x$ 是葉子節點,則此屬性值爲 $\infty$
\item min\text{-}val:以 $x$ 爲根的子樹中的最小值。
\item max\text{-}val:以 $x$ 爲根的子樹中的最大值。
\stopigBase

計算方式如下:
\startformula
x.min\text{-}val = \startmathcases
\NC x.left.min\text{-}val \NC \text{如果 $x.left \ne T.nil$,} \NR
\NC x.key \NC \text{否則。} \NR
\stopmathcases
\stopformula

\startformula
x.max\text{-}val = \startmathcases
\NC x.right.max\text{-}val \NC \text{如果 $x.right \ne T.nil$,} \NR
\NC x.key \NC \text{否則。} \NR
\stopmathcases
\stopformula

\startsplitformula\startmathalignment
\NC x.min\text{-}gap = \min\{ \NC
  x.left.min\text{-}gap, \NR
\NC \NC x.right.min\text{-}gap, \NR
\NC \NC x.key - x.left.max\text{-}val, \NR
\NC \NC x.right.min\text{-}val - x.key \} \NR
\stopmathalignment\stopsplitformula


根據定理 17.1,插入、刪除操作可以在 $O(\lg n)$ 時間內完成,
而更新這幾個屬性均可在旋轉操作時用 $O(1)$ 時間完成。

\ALGO{MIN-GAP} 只需返回根節點的 $min\text{-}gap$ 屬性值,運行時間爲 $O(1)$。
\stopANSWER
