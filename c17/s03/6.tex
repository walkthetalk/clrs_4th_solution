\startEXERCISE\DIFFICULT
VLSI 數據庫通常用一組矩形表示一塊集成電路,
假設每個矩形的邊都平行於 x 軸或者 y 軸,
這樣可以用矩形的最小、最大 x 軸 y 軸座標來表示一個矩形。
請給出一個時間爲 $O(n\lg n)$ 的算法,
來確定 $n$ 個這種矩形中是否存在兩個重疊的矩形。
你的算法無需要輸出所有重疊的矩形,
但如果兩個矩形完全重疊,即使邊界線不想交,也要給出正確結果。
(\hint 移動一條“掃描”線,穿過所有的矩形。)
\stopEXERCISE

\startANSWER
預處理。先按 $y$ 座標從小到大排序,每個矩形會出現兩次,
用以確定掃描順序,需要時間 $O(n\lg n)$。

掃描。如果掃到的 $y$ 是矩形的上邊界(較小的 $y$),
則檢查該矩形的 $x$ 區間是否與區間樹中已有區間重疊,
如果沒有重疊,就插入區間樹。
而如果掃到了矩形的下邊界,則將其在區間樹中刪除。
有 $n$ 個節點的區間樹上執行 $2n$ 次查找、插入、刪除,
所需時間 $O(n\lg n)$。

總時間爲 $O(n\lg n)$。
\stopANSWER
