\startEXERCISE
有二叉搜索樹 $T_1$,通過一系列右旋操作變成二叉搜索樹 $T_2$,
則稱 $T_1$ 可以\emph{右轉(right-converted)}成 $T_2$。
試舉反例,描述 $T_1$ 不能右轉成 $T_2$ 的情況。
然後證明:如果 $T_1$ 可以右轉成 $T_2$,
那麼可以通過 $O(n^2)$ 次右旋來實現。
\stopEXERCISE

\startANSWER
\startcombination[2*1]
{\externalfigure[output/e13_2_5-1]}{}
{\externalfigure[output/e13_2_5-2]}{}
\stopcombination

$T_1$ 中右孩子都不可能變成根節點。
所以新的根節點只能是 $T_1$ 的根節點到最小節點路徑上的節點。
我們可以在 $O(n)$ 時間解決根節點,然後遞迴解決左右子樹,時間複雜度爲:
$T(n)=T(n_L)+T(n_R)+O(n)=O(n^2)$,
其中 $n_L+n_R=n-1$。
\stopANSWER
