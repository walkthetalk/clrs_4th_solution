\startEXERCISE
如果 \ALGO{TREE-DELETE} 中的節點 $z$ 有兩個子節點,
我們可以選擇其前驅作爲 $y$ 而不是後繼。
如果這樣做,我們還需要對 \ALGO{TREE-DELETE} 做哪些修改?
有些人提出了一個公平策略,
即賦予前驅和後繼相同的優先級,
從而得到了更好的性能。
如何修改 \ALGO{TREE-DELETE} 來實現這樣的策略?
\stopEXERCISE

\startANSWER
\CLRSH{TREE-DELETE(T, z)}
\startCLRSCODE
if z.left == NIL
	\ALGO{TRANSPLANT(T, z, z.right)}
else if z.right == NIL
	\ALGO{TRANSPLANT(T, z, z.left)}
else
	flag = \ALGO{RANDOM()}		// 0 or 1
	if flag == 0
		y = \ALGO{TREE-MAXIMUM(z.left)}
		if y != z.left
			\ALGO{TRANSPLANT(T, y, y.left)}
			y.left = z.left
			y.left.p = y
		\ALGO{TRANSPLANT(T, z, y)}
		y.right = z.right
		y.right.p = y
	else
		y = \ALGO{TREE-MINIMUM(z.right)}
		if y != z.right
			\ALGO{TRANSPLANT(T, y, y.right)}
			y.right = z.right
			y.right.p = y
		\ALGO{TRANSPLANT(T, z, y)}
		y.left = z.left
		y.left.p = y
\stopCLRSCODE
\stopANSWER
