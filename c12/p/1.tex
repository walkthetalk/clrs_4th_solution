\startPROBLEM
(Binary search trees with equal keys)
相同的關鍵字給二叉搜索樹的實現帶來一個難題。
% a
\startigBase[a]\startitem
二叉搜索樹初始爲空,
當用 \ALGO{TREE-INSERT} 插入 $n$ 個具有相同關鍵字的數據,
其漸進性能是多少?
\stopitem\stopigBase

\startANSWER
最壞情況下會形成一個只有右子樹的鏈,
漸進性能爲 $\sum_{i=1}^{n}i = O(n^2)$。
\stopANSWER

我們建議通過以下方式改進 \ALGO{TREE-INSERT}:
在第 5 行前比較 $z.key == x.key$,
在第 11 行前比較 $z.key == y.key$。
如果比較結果相等,我們根據以下策略之一來實現。
對於每種策略,找到上一項的答案。
(對於第 5 行,策略是比較 $z$ 和 $x$,
而對於第 11 行,是比較 $z$ 和 $y$)

% b
\startigBase[continue]\startitem
在節點 $x$ 中維護一個布爾標志 $x.b$,
並根據 $x.b$ 的值將 $x$ 設置成 $x.left$ 或者 $x.right$,
這樣當插入的關鍵字與 $x$ 相同時,
 $x.b$ 的值就會在 FALSE 和 TRUE 間來回切換。
\stopitem\stopigBase

\startANSWER
所構造的是平衡二叉樹,
所構造的兩棵子樹大小最多差 1。
高度爲 $\Theta(\lg{n})$。
時間爲 $\sum_{i=1}^{n}\lg{n} = \Theta(\lg{n})$。
\stopANSWER

% c
\startigBase[continue]\startitem
在 $x$ 中維護一個鏈表,
存儲關鍵字與 $x$ 相同的節點,
將 $z$ 插入此鏈表。
\stopitem\stopigBase

\startANSWER
樹的高度將是 0,
單次插入所花時間爲 $\Theta(1)$。
總時間爲 $\Theta(n)$。
\stopANSWER

% d
\startigBase[continue]\startitem
隨機地將 $x$ 設置成 $x.left$ 或 $x.right$。
(給出最壞情況的性能,並推測期望運行時間)
\stopitem\stopigBase

\startANSWER
所謂最壞情況,就是隨機的結果始終如一,
全部爲左或全部爲右。
結果與第一項一樣,時間爲 $\Theta(n^2)$。

接下來看期望運行時間。
隨機選擇的結果是幾乎一半爲左,另一半爲右,
也就是說說生成的是平衡二叉樹,
高度爲 $\lg n$,
期望運行時間爲 $n\lg n$。
\stopANSWER

\stopPROBLEM
