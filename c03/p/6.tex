\startPROBLEM
(\m{O}與\m{\Omega}的一些變形)
某些作者定義 $\Omega$ 的方式跟本書有點差異;
我們使用 \m{\mathop{\Omega}\limits^{\infty}} 來表示這種定義。
若存在正常數 $c$,
使得存在無限個正整數 $n$,
滿足 $f(n)\ge c g(n)\ge 0$,
則稱 \m{f(n) = \mathop{\Omega}\limits^{\infty}(g(n))}。
\startigBase[a]
% a
\startitem
證明:對任意兩個漸進非負函式 $f(n)$ 和 $g(n)$,
肯定滿足 $f(n) = O(g(n))$ 或 $f(n) = \mathop{\Omega}\limits^{\infty}(g(n))$ 中的一個,
也可能二者均成立。
然而,如果用 $\Omega$ 代替 $\mathop{\Omega}\limits^{\infty}$,
那麼該命題卻不成立。

\startANSWER
我們需要比較 $cg(n) \le f(n)$;
如果存在無限個正整數 $n$ 能使其成立,
則有 \m{\mathop{\Omega}\limits^{\infty}};
反之,則令滿足此式的 $n$ 最大爲 $n_0$,有:
\startformula
\forall n > n_0: f(n) < cg(n)
\stopformula
足以說明 $f(n) = O(g(n))$。

如果 $f(n) = g(n)$,顯然兩式均成立。

但是對於 $\Omega$,卻不一定,
如 $n = \mathop{\Omega}\limits^{\infty}(n^{\sin{n}})$,
但 $n \ne \Omega(n^{\sin{n}})$。

再如, $f(n)=n^2(n\mod 2)$, $g(n)=n$,
對於所有偶數, $g(n)$ 更大,
但對於所有奇數, $f(n)$ 增長的更快。
\stopANSWER
\stopitem

% b
\startitem
是否存在兩個非負函式 $f(n)$ 和 $g(n)$,
既不滿足 $f(n)=O(g(n))$ 也不滿足 $f(n)=\Omega(g(n))$?
若有,請舉例。

\startANSWER
$f(n)=n^2(n\mod 2)$, $g(n)=n$,
對於所有偶數, $g(n)$ 更大,
但對於所有奇數, $f(n)$ 增長的更快。
\stopANSWER
\stopitem

% c
\startitem
刻畫程序運行時間時,
用 $\mathop{\Omega}\limits^{\infty}$ 代替 $\Omega$ 有
哪些潛在優點和缺點。

\startANSWER
優點:你可以使用第一問中所描述的性質。

劣勢:所得到的集合中雖然有無限個數,但其值可能很稀疏。
而且對集合外的數沒有任何約束。
\stopANSWER
\stopitem

\stopigBase

某些作者定義 $O$ 的方式也跟本書有點差異;
用 $O'$ 來表示這種定義。
當且儘當 $|f(n)| = O(g(n))$ 時,我們稱 $f(n) = O'(g(n))$。

\startigBase[a,continue]
% d
\startitem
如果將 $O$ 改爲 $O'$,但 $\Omega$ 保持不變,
定理 3.1 中的“當且儘當”兩個方向上各自會出現什麼情況?
附定理 3.1:

任意兩個函式 $f(n)$ 和 $g(n)$,
當且僅當 $f(n)=O(g(n))$ 及 $f(n)=\Omega(g(n))$ 時,
有 $f(N)=\Theta(g(n))$。

\startANSWER
定理 3.1 中的“當且儘當”要改爲“蘊含”,即 \m{\Theta \Rightarrow O'},
反向則不成立。
例如:

函數 $f(n) = n \cdot \sin{n}$,
即爲 $O'(n)$,但卻不是 $O(n)$ 或 $\Theta(n)$。
\stopANSWER
\stopitem
\stopigBase

有些作者給 $O$ 添上對數因子後改爲 $\tilde{O}$:
\startformula
\tilde{O} = \lbrace f(n)
 : \exists c > 0, k > 0, n_0 > 0
\forall n \ge n_0
 : 0 \le f(n) \le c g(n) \lg^k(n) \rbrace
\stopformula
\startigBase[a,continue]
% e
\startitem
按類似的方式定義 $\tilde{\Omega}$ 和 $\tilde{\Theta}$,
並證明:

任意兩個函式 $f(n)$ 和 $g(n)$,
當且僅當 $f(n)=\tilde{O}(g(n))$ 及 $f(n)=\tilde{\Omega}(g(n))$ 時,
有 $f(N)=\tilde{\Theta}(g(n))$。

\startANSWER
\startsplitformula\startmathalignment
\NC \tilde{\Omega} = \lbrace f(n): \NC
  \exists c, k, n_0>0 \NR
\NC \NC \forall n > n_0
  : 0 \leq cg(n) \lg^{-k}(n) \leq f(n) \rbrace \NR

\NC \tilde{\Theta} = \lbrace f(n): \NC
 \exists c_1, c_2, k_1, k_2, n_0 > 0 \NR
\NC \NC \forall n > n_0
 : 0 \leq c_1g(n) \lg^{-k_1}(n) \leq f(n) \leq c_2g(n) \lg^{k_2}(n)\rbrace \NR
\stopmathalignment\stopsplitformula

$\tilde{\Theta}$ 顯然蘊含 $\tilde{O}$ 和 $\tilde{\Omega}$。
反之,如果:
\startsplitformula\startmathalignment
\NC \tilde{\Omega} = \lbrace f(n): \NC
  \exists c_1, k_1, n_1>0 \NR
\NC \NC \forall n > n_1
  : 0 \leq c_1 g(n) \lg^{-k_1}(n) \leq f(n) \rbrace \NR

\NC \tilde{O} = \lbrace f(n): \NC
 \exists c_2, k_2, n_2 > 0 \NR
\NC \NC \forall n \ge n_2
 : 0 \le f(n) \le c_2 g(n) \lg^{k_2}(n) \rbrace \NR
\stopmathalignment\stopsplitformula
那麼:
\startsplitformula\startmathalignment
\NC \tilde{\Theta} = \lbrace f(n): \NC
 \exists c_1,c_2,k_1,k_2 > 0, n_0 > \max(n_1,n_2) \NR
\NC \NC \forall n \ge n_0
 : 0 \le c_1 g(n) \lg^{-k_1}(n) \le f(n) \le c_2 g(n) \lg^{k_2}(n) \rbrace \NR
\stopmathalignment\stopsplitformula
\stopANSWER
\stopitem
\stopigBase

\stopPROBLEM
