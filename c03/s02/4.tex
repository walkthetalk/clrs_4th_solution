\startEXERCISE
證明定理 3.1。

當且僅當 \m{f(n) = O(g(n))} 並且 \m{f(n) = \Omega(g(n))} 時,才有 \m{f(n) = \Theta(g(n))}。
\stopEXERCISE
\startANSWER
如果 $f(n) = \Theta(g(n))$,則:
\startformula
0 \le c_1g(n) \le f(n) \le c_2g(n) \quad \text{若} n > n_0
\stopformula
將兩個常數 $c_1$ 和 $c_2$ 代入 $O$ 和 $\Omega$ 的定義中即得:
\startsplitformula\startalign
\NC f(n) \NC = O(g(n)) \NR
\NC f(n) \NC = \Omega(g(n)) \NR
\stopalign\stopsplitformula

反之,如果 $f(n) = O(g(n))$ 並且 $f(n) = \Omega(g(n))$,則:
\startsplitformula\startalign
\NC 0 \le c_3 g(n) \le f(n) \NC \quad \text{若} n \ge n_1 \NR
\NC 0 \le f(n) \le c_4 g(n) \NC \quad \text{若} n \ge n_2 \NR
\stopalign\stopsplitformula
設 \m{n_3=\max(n_1,n_2)},合並兩個不等式,得:
\startformula
0 \leq c_3g(n) \leq f(n) \leq c_4g(n) \quad \text{若} n > n_3
\stopformula
\stopANSWER
