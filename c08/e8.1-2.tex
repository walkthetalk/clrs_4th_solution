\startEXERCISE[exercise:bound_of_lgnfrac]
不用 Stirling 近似公式,給出 $\lg(n!)$ 的漸進緊確界。
利用 A.2 中介紹的技術來求累加和 $\sum_{k=1}^n\lg{k}$。
\stopEXERCISE

\startANSWER
先來證明解爲 $O(n\lg{n})$:
\startformula
\sum_{k=1}^n\lg{k} \le \sum_{k=1}^n\lg{n} = n\lg{n} = O(n\lg{n})
\stopformula

再來證明解爲 \m{\Omega(n\lg{n})}:
\startsplitformula\startmathalignment
\NC \sum_{k=1}^n\lg{k}
   \NC= \sum_{k=1}^{\lfloor n/2 \rfloor}\lg{k} +
	\sum_{k=\lfloor n/2 \rfloor + 1}^n\lg{k} \NR
\NC\NC\ge \sum_{k=\lfloor n/2 \rfloor + 1}^n\lg{k} \NR
\NC\NC\ge \sum_{k=\lfloor n/2 \rfloor + 1}^n\lg{\frac{n}{2}} \NR
\NC\NC\ge (\frac{n}{2}-1)\lg{\frac{n}{2}} \NR
\NC\NC\ge (\frac{n}{2}-1)(\lg{n}-1) \NR
\NC\NC= \Omega(n\lg{n}) \NR
\stopmathalignment\stopsplitformula
\stopANSWER
