\startEXERCISE[exercise:bound_of_lgnfrac]
不用 Stirling 近似公式,給出 $\lg(n!)$ 的漸進緊確界。
利用 A.2 中介紹的技術來求累加和 $\sum_{k=1}^n\lg{k}$。
\stopEXERCISE

\startANSWER
根據公式 A.18:
\startsplitformula\startmathalignment[n=3,align={right,middle,left}]
\NC \int_0^n \lg(x)dx \NC \le \sum_{k=1}^{n}\lg(k) \le \NC \int_1^{n+1}\lg(x)dx \NR
\NC x\lg(x) - x|_0^n  \NC \le \sum_{k=1}^{n}\lg(k) \le \NC x\lg(x) - x|_1^{n+1} \NR
\NC n\lg n - n        \NC \le \sum_{k=1}^{n}\lg(k) \le \NC (n+1)\lg(n+1) - n    \NR
\stopmathalignment\stopsplitformula
所以結果爲 $\Theta(n\lg n)$。
\stopANSWER
