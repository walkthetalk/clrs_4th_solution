\startPROBLEM
(Mergeable heaps using linked lists)
\emph{可合並堆}(\emph{mergeable heap})支持以下操作:
 \ALGO{MAKE-HEAP}(創建一個空的可合並堆)、
 \ALGO{INSERT}、 \ALGO{MINIMUM}、 \ALGO{EXTRACT-MIN} 和 \ALGO{UNION}
\footnote{%
由於我們定義的是 \ALGO{MINIMUM} 和 \ALGO{EXTRACT-MIN},
因此我們也可以將其稱爲\emph{可合並最小堆};
如果我們定義的是 \ALGO{MAXIMUM} 和 \ALGO{EXTRACT-MAX},
則可以將其稱爲\emph{可合並最大堆}。
}。
試說明下列各種情況下如何用鏈表實現可合並堆。
試着使各種操作盡可能高效。
根據動態集合的規模 $s$ 分析每種操作的運行時間。
\startigBase[a]
\item 鏈表已排序;
\item 鏈表未排序;
\item 鏈表未排序,且待合並的動態集合是不相交的。
\stopigBase
\stopPROBLEM

\startANSWER
\scomponent{tbl10-2.tex}

對於未排序鏈表,所需時間與是否有相同元素無關。
\stopANSWER
