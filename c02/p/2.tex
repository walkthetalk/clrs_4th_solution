\startPROBLEM
冒泡排序的正確性

冒泡排序通俗易懂,但效率不高。
他通過重複交換兩個(次序顛倒的)相鄰元素的方式進行排序。
例程 \ALGO{BUBLLE-SORT} 對數列 $A[1:n]$ 進行排序。

\CLRSH{BUBBLE-SORT(A, n)}
\startCLRSCODE
for i = 1 to n - 1
	for j = n downto i + 1
		if A[j] < A[j - 1]
			\ALGO{SWAP(A[j],A[j - 1])}
\stopCLRSCODE

\startigBase[a]
\item 在數列 $A$ 上執行 \ALGO{BUBBLE-SORT(A,n)} 後變爲數列 $A'$,
要證明算法的正確性,
除了要證明
\startformula
A'[1]\le A'[2]\le \cdots \le A'[n]
\stopformula
之外,還需要證明什麼?
\stopigBase

\startANSWER
還需要證明排序完成後,新數列包含舊數列中的所有元素。
\stopANSWER

\startigBase[continue]
\item 分析並證明內層 \CLRSCODE{for} 循環的循環不變性。
要求使用本章所展示的證明方式。
\stopigBase

\startANSWER
\TODO{略。}
\stopANSWER

\startigBase[continue]
\item 利用上一問所證明的循環不變性的終止條件,
分析外層循環的循環不變性,並以此證明不等式(2.5)。
要求使用本章所展示的證明方式。
\stopigBase

\startANSWER
\TODO{略。}
\stopANSWER

\startigBase[continue]
\item 冒泡排序在最差情況下運行時間怎樣?
與插入排序相比如何?
\stopigBase

\startANSWER
冒泡排序的比較次數最多爲:$\sum_{i=1}^{n-1}{n-i} = \frac{n(n - 1)}{2}$,
相應的交換次數也一樣,所以最壞情況下所用時間爲 $\Theta(n^2)$,
與插入排序所用時間相同。

通常,兩種算法最好情況下時間均爲 $\Theta(n)$,
但是此處的實現卻爲 $\Theta(n^2)$。
要想在最好情況下達到 $\Theta(n)$,
在外部循環中,如果沒有發生任何交換就直接返回。

另外,冒泡排序會比插入排序慢很多,
因爲交換所引入的賦值操作太多了。
\stopANSWER

\stopPROBLEM
