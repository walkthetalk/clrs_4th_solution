\startPROBLEM
现有多项式:
\startsplitformula\startmathalignment
\NC P(x) \NC = \sum_{k=0}^{n} a_kx^k \NR
\NC \NC = a_0 + a_1 x + a_2 x^2 + \cdots + a_{n-1}x^{n-1} + a_n x^n \NR
\stopmathalignment\stopsplitformula
给定系数 $a_0,a_1,a_2,\cdots,a_n$ 后,
可以按 Horner 定律計算多項式的值:
\startformula
P(x) = a_0 + x(a_1 + x(a_2 + \cdots + x(a_{n-1} + xa_n) \cdots))
\stopformula
具體實現如下:

\CLRSH{HORNER(A, n, x)}
\startCLRSCODE
p = 0
for i = n downto 0
	p = A[i] + x \cdot p
return p
\stopCLRSCODE

\startigBase[a]
\item 用 $\Theta$ 表示此例程的運行時間。

\startANSWER
$\Theta(n)$
\stopANSWER

\item 按此多項式的原始定義來計算,需要多少時間?

\startANSWER
\startCLRSCODE
p = 0
for i = 0 to n
	m = 1
	for k = 1 to i
		m = m \cdot x
	p = p + A[i] \cdot m
\stopCLRSCODE
所用時間爲 $\Theta(n^2)$。
\stopANSWER

\item 分析 \ALGO{HORNER} 的循環不變性。
每次迭代前:
\startformula
y = \sum_{k=0}^{n-(i+1)} A_{k+i+1}\cdot x^k
\stopformula
將空多項式求和結果定義爲 0。
以本章循環不變性的證明方式,
證明最終結果爲:
\startformula
y = \sum_{k=0}^n a_kx^k
\stopformula

\startANSWER
每次迭代完成後:
\startsplitformula\startalign
\NC p \NC = A_i + x \sum_{k=0}^{n-(i+1)}A_{k+i+1}x^k \NR
\NC	\NC = A_i x^0 + \sum_{k=0}^{n-i-1}A_{k+i+1}x^{k+1} \NR
\NC	\NC = A_i x^0 + \sum_{k=1}^{n-i}A_{k+i}x^{k} \NR
\NC	\NC = \sum_{k=0}^{n-i}A_{k+i}x^{k} \NR
\stopalign\stopsplitformula
最後一次 $i$ 爲 $0$,
代入上式可得 $p = \sum_{k=0}^{n}a_{k}x^{k}$。
\stopANSWER

\stopigBase

\stopPROBLEM
