\startEXERCISE
以遞迴方式進行插入排序:
要對 $A[1:n]$ 排序,
可以先對子數列 $A[1:n-1]$ 排序,
然後將 $A[n]$ 插入子數列 $A[1:n-1]$ 中。
試給出此算法的僞碼。
並分析最壞情況下的算法是如何運行的。
\stopEXERCISE
\startANSWER
\CLRSH{INSERTION_SORT_R(A, n)}
\startCLRSCODE
if n < 2
	return
\ALGO{INSERTION_SORT_R(A, n-1)}
key = A[n]
i = n - 1
while i \ge 0 and A[i] > key
	A[i+1] = A[i]
	i = i - 1
A[i + 1] = key
\stopCLRSCODE

\startformula
T(n) = \startmathcases
\NC 0 \NC \text{若 $n = 1$,} \NR
\NC T(n-1) + C(n-1)	\NC \text{否則。} \NR
\stopmathcases
\stopformula

初始數列爲逆序的情況下,需要移動元素最多,每次遞迴都需要遍歷所有元素:

$T(n) = 1 + 2 + 3 + ... + (n-1) = \Theta(n^2)$
\stopANSWER
